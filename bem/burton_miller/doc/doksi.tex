\documentclass[a4paper, 10pt]{article}

\usepackage[utf8]{inputenc}
\usepackage{a4wide}
\usepackage{amsmath}
\usepackage{amssymb}
\usepackage{bm}

\newcommand{\te}{\mathrm{e}}
\newcommand{\ti}{\mathrm{i}}
\newcommand{\td}{\mathrm{d}}

\newcommand{\tn}{\bm{\mathnormal{n}}}
\newcommand{\tr}{\bm{\mathnormal{r}}}
\newcommand{\tx}{\bm{\mathnormal{x}}}
\newcommand{\tT}{^{\mathrm{T}}}
\newcommand{\ty}{\bm{\mathnormal{y}}}
\newcommand{\sx}{\bm{\mathnormal{x}}}
\newcommand{\sy}{\bm{\mathnormal{y}}}
\newcommand{\sn}{\bm{\mathnormal{n}}}
\newcommand{\bN}{\mathbf{N}}
\newcommand{\bp}{\mathbf{p}}
\newcommand{\bq}{\mathbf{q}}
\newcommand{\bH}{\mathbf{H}}
\newcommand{\bG}{\mathbf{G}}


\author{Péter Rucz}
\title{Burton-Miller}

\begin{document}

\maketitle

\section{The Green function and its derivatives}

The Green-function reads as
%
\begin{equation}
	G_k(\tx, \ty) = \frac{\te^{-\ti k r}}{4 \pi r}, \quad
	r = |\ty - \tx|
\end{equation}

\begin{align}
	\frac{\partial G_k(\tx, \ty)}{\partial \tn_y}
	&= -G_k(\tx, \ty) \frac{1 + \ti kr}{r}\frac{\partial r}{\partial \tn_y} \qquad \text{with} \qquad \frac{\partial r}{\partial \tn_y} = \frac{\tr \cdot \tn_y}{r} \\
	\frac{\partial G_k(\tx, \ty)}{\partial \tn_x}
	&= -G_k(\tx, \ty) \frac{1 + \ti kr}{r}\frac{\partial r}{\partial \tn_x} \qquad \text{with} \qquad \frac{\partial r}{\partial \tn_x} = -\frac{\tr \cdot \tn_x}{r}
\end{align}

\begin{align}
	\frac{\partial^2 G_k(\tx, \ty)}{\partial \tn_x \partial \tn_y}
	&= \frac{\te^{- \ti kr}}{4 \pi r^3}\left[\left(3 + 3 \ti kr - k^2 r^2\right) \frac{\partial r}{\partial \tn_x}\frac{\partial r}{\partial \tn_y} + (1 + \ti kr) \tn_x \cdot \tn_y \right] \\
	&= G_k(\tx,\ty)\left[\left(\frac{3}{r^2} + \frac{3\ti k}{r} - k^2\right) \frac{\partial r}{\partial \tn_x}\frac{\partial r}{\partial \tn_y}
	+ \frac{1}{r^2}(1 + \ti kr) \tn_x \cdot \tn_y \right]
\end{align}

The static part is given as:
%
\begin{equation}
	G_0(\tx, \ty) = \frac{1}{4\pi r}
\end{equation}

\begin{equation}
	\frac{\partial G_0(\tx, \ty)}{\partial \tn_x} = -\frac{G_0(\tx, \ty)}{r} \frac{\partial r}{\partial \tn_x}
\end{equation}

\begin{equation}
	\frac{\partial G_0(\tx, \ty)}{\partial \tn_x \partial \tn_y} = \frac{G_0(\tx, \ty)}{r^2} \left[3 \frac{\partial r}{\partial \tn_x}\frac{\partial r}{\partial \tn_y} + \tn_x \cdot \tn_y \right]
\end{equation}

Taking the difference of the second derivatives:
%
\begin{multline}
	\frac{\partial G_k(\tx, \ty)}{\partial \tn_x \partial \tn_y} - \frac{\partial G_0(\tx, \ty)}{\partial \tn_x \partial \tn_y} = \\
	\frac{\te^{- \ti kr}}{4 \pi r^3}\left[\left(3 \cdot (1 - \te^{\ti kr}) + 3 \ti kr - k^2 r^2\right) \frac{\partial r}{\partial \tn_x}\frac{\partial r}{\partial \tn_y} + (1 - \te^{\ti kr} + \ti kr) \tn_x \cdot \tn_y \right]
\end{multline}
%
Taking the limit $r \rightarrow 0$, the difference can be expressed as a Taylor-series
%
\begin{equation}
	\lim\limits_{\sy \rightarrow \sx} \left(\frac{\partial G_k(\tx, \ty)}{\partial \tn_x \partial \tn_y} - \frac{\partial G_0(\tx, \ty)}{\partial \tn_x \partial \tn_y}\right) = \frac{k^2}{8 \pi r} + \frac{\ti k^3}{12 \pi} + o(r)
\end{equation}

\section{The Burton-Miller formulation}

\begin{equation}
	C(\sx) p (\sx) = \int_{\Gamma}\left[\frac{\partial{G}_k (\sx, \sy)}{\partial \sn_y} p(\sy)
	- G_k(\sx,\sy)\frac{\partial p(\sy)}{\partial \sn_y}\right] \td \Gamma_y + p_{in}(\sx)
\end{equation}

The normal derivative ($ \partial / \partial \sn_x$):

\begin{equation}
	C(\sx) \frac{\partial p(\sx)}{\partial \sn_x}
	=
	\int_{\Gamma}\left[\frac{\partial^2{G}_k (\sx, \sy)}{\partial \sn_y \partial \sn_x} p(\sy) - \frac{\partial G_k(\sx,\sy)}{\partial \sn_x}\frac{\partial p(\sy)}{\partial \sn_y}\right] \td \Gamma_y + \frac{\partial p_{in}(\sx)}{\partial \sn_x}
\end{equation}

Using the Burton-Miller formulation for surface points ($C(\sx) = 1/2$), the linear combination is calculated with the coupling constant $\alpha$

\begin{multline}
	C(\sx) p (\sx) - p_{in}(\sx)
	+
	\alpha C(\sx) \frac{\partial p(\sx)}{\partial \sn_x}
	-
	\alpha \frac{\partial p_{in}(\sx)}{\partial \sn_x}
	= \\
	\int_{\Gamma}\left[\frac{\partial{G}_k (\sx, \sy)}{\partial \sn_y} p(\sy)
	- G_k(\sx,\sy)\frac{\partial p(\sy)}{\partial \sn_y}\right] \td \Gamma_y \\
	+
	\alpha \int_{\Gamma}
	\left[
	\frac{\partial^2{G}_k (\sx, \sy)}{\partial \sn_y \partial \sn_x} p(\sy)
	- \frac{\partial G_k(\sx,\sy)}{\partial \sn_x}\frac{\partial p(\sy)}{\partial \sn_y}
	\right] \td \Gamma_y
\end{multline}

Using shape functions $N_i(\sx)$ for the pressure and its normal derivative:

\begin{align}
	p(\sx) & = \sum\limits_{i=1}^P N_i(\sx)p_i \\
	\frac{\partial p(\sx)}{\partial \sn_x} & = q(\sx) = \sum\limits_{i=1}^P N_i(\sx)q_i
\end{align}

Rearranging:

\begin{multline}
	\frac{1}{2} p (\sx)
	-
	p_{in}(\sx)
	+
	\frac{\alpha}{2} q(\sx)
	-
	\alpha q_{in}(\sx)
	= \\
	\left(
	\int_{\Gamma} \frac{\partial{G}_k (\sx, \sy)}{\partial \sn_y} p(\sy) \td \Gamma_y
	+
	\alpha \int_{\Gamma}
	\frac{\partial^2{G}_k (\sx, \sy)}{\partial \sn_y \partial \sn_x} p(\sy) \td \Gamma_y
	\right)
	\\
	-
	\left(
	\int_{\Gamma} G_k(\sx,\sy) q(\sy) \td \Gamma_y
	+
	\alpha \int_{\Gamma}
	\frac{\partial G_k(\sx,\sy)}{\partial \sn_x}q(\sy) \td \Gamma_y
	\right)
\end{multline}

In matrix form:

\begin{align}
	\frac{1}{2} \mathbf{p} - \mathbf{p}_{in} + \frac{\alpha}{2} \mathbf{q} - \alpha \mathbf{q}_{in} = \bH \bp - \bG \bq
\end{align}

With

\begin{align}
	\bH_{ij}
	& = \int_{\Gamma}\frac{\partial{G}_k (\sx_i, \sy)}{\partial \sn_y} N_j(\sy) \td \Gamma(\sy)
	+ \alpha \int_\Gamma \frac{\partial^2{G}_k (\sx_i, \sy)}{\partial \sn_y \partial \sn_x} N_j(\sy) \td \Gamma(\sy) \nonumber \\
	\bG_{ij}
	& = \int_{\Gamma} G_k(\sx_i,\sy) N_j(\sy) \td \Gamma(\sy)
	+ \alpha \int_\Gamma \frac{\partial{G}_k (\sx_i, \sy)}{\partial \sn_x} N_j(\sy) \td \Gamma(\sy)
\end{align}




\section{Treatment of hypersingular integrals}

Using the Burton-Miller formulation for surface points ($C(\sx) = 1/2$), the linear combination is calculated with the coupling constant $\alpha$
%
\begin{multline}
	\alpha \int_\Gamma \frac{\partial^2{G}_k (\sx, \sy)}{\partial \sn_y \partial \sn_x} p(\sy) \td \Gamma_y -
	\alpha \int_\Gamma \frac{\partial{G}_k (\sx, \sy)}{\partial \sn_x} \frac{\partial p(\sy)}{\partial \sn_y} \td \Gamma_y = \\
	 = \int_{\Gamma}\left[-\frac{\partial{G}_k (\sx, \sy)}{\partial \sn_y} p(\sy) + G_k(\sx,\sy)\frac{\partial p(\sy)}{\partial \sn_y}\right] \td \Gamma_y \\
	- \frac{1}{2} p(\sx) - \frac{1}{2}\alpha \frac{ \partial p(\sx)}{\partial \sn_x} + p_{in}(\sx) + \alpha \frac{\partial p_{in}(\sx)}{\partial \sn_x}
\end{multline}

\begin{align}
	\int_\Gamma \frac{\partial G_0(\sx, \sy)}{\partial \sn_y} \td \Gamma_y & = 0 \\
	\int_\Gamma \frac{\partial^2 G_0(\sx, \sy)}{\partial \sn_x \partial \sn_y} \td \Gamma_y & = 0 \\
	\int_\Gamma \frac{\partial^2 G_0(\sx, \sy)}{\partial \sn_x \partial \sn_y}(\sy - \sx) \td \Gamma_y & = \int_\Gamma \frac{\partial G_0(\sx, \sy)}{\partial \sn_x} \sn_y \td \Gamma_y 
\end{align}




\end{document}