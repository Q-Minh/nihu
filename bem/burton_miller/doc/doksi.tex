\documentclass[a4paper, 10pt]{article}

\usepackage[utf8]{inputenc}
\usepackage{a4wide}
\usepackage{amsmath}
\usepackage{amssymb}
\usepackage{bm}

\newcommand{\te}{\mathrm{e}}
\newcommand{\ti}{\mathrm{i}}
\newcommand{\td}{\mathrm{d}}

\newcommand{\tn}{\bm{\mathnormal{n}}}
\newcommand{\tr}{\bm{\mathnormal{r}}}
\newcommand{\tx}{\bm{\mathnormal{x}}}
\newcommand{\tT}{^{\mathrm{T}}}
\newcommand{\ty}{\bm{\mathnormal{y}}}
\newcommand{\sx}{\bm{\mathnormal{x}}}
\newcommand{\sy}{\bm{\mathnormal{y}}}
\newcommand{\sn}{\bm{\mathnormal{n}}}
\newcommand{\bN}{\mathbf{N}}
\newcommand{\bp}{\mathbf{p}}
\newcommand{\bq}{\mathbf{q}}
\newcommand{\bH}{\mathbf{H}}
\newcommand{\bG}{\mathbf{G}}


\author{Péter Rucz}
\title{Burton-Miller}

\begin{document}

\maketitle

\section{The Green function and its derivatives}

The Green-function reads as

\begin{equation}
	G_k(\tx, \ty) = \frac{\te^{-\ti k r}}{4 \pi r}
\end{equation}

\begin{equation}
	\frac{\partial G_k}{\partial n(\tx)} = -\frac{\te^{-\ti k r}}{4 \pi r^2}(1 + \ti kr)\frac{\partial r}{\partial \tn(\tx)} \qquad \text{with} \qquad \frac{\partial r}{\partial \tn(\tx)} = \frac{\tr \cdot \tn(\tx)}{r}
\end{equation}

%(exp(-k*r*i)*(- k^2*r^2 + k*r*(2*i) + 2))/(4*pi*r^3)

\begin{equation}
	\frac{\partial G_k(\tx, \ty)}{\partial \tn(\tx) \partial \tn(\ty)} = \frac{\te^{- \ti kr}}{4 \pi r^3}\left[(3 + 3 \ti kr - k^2 r^2) \frac{\partial r}{\partial \tn(\tx)}\frac{\partial r}{\partial \tn(\ty)} + (1 + \ti kr) \tn(\tx) \cdot \tn (\ty) \right]
\end{equation}

The static part is given as:

\begin{equation}
	G_0(\tx, \ty) = \frac{1}{4\pi r}
\end{equation}

\begin{equation}
	\frac{\partial G_0(\tx, \ty)}{\partial \tn(\tx)} = -\frac{1}{4\pi r^2} \frac{\partial r}{\partial \tn(\tx)}
\end{equation}

\begin{equation}
	\frac{\partial G_0(\tx, \ty)}{\partial \tn(\tx) \partial \tn(\ty)} = \frac{1}{4 \pi r^3} \left[3 \frac{\partial r}{\partial \tn(\tx)}\frac{\partial r}{\partial \tn(\ty)} + \tn(\tx) \cdot \tn (\ty) \right]
\end{equation}

Taking the difference of the second derivatives:

\begin{align}
	\frac{\partial G_k(\tx, \ty)}{\partial \tn(\tx) \partial \tn(\ty)} - \frac{\partial G_0(\tx, \ty)}{\partial \tn(\tx) \partial \tn(\ty)} = 
	\frac{\te^{- \ti kr}}{4 \pi r^3}\left[3 \cdot (1 - \te^{\ti kr}) + 3 \ti kr - k^2 r^2) \frac{\partial r}{\partial \tn(\tx)}\frac{\partial r}{\partial \tn(\ty)} + (1 - \te^{\ti kr} + \ti kr) \tn(\tx) \cdot \tn (\ty) \right]
\end{align}

Taking the limit $r \rightarrow 0$, the difference can be expressed as a Taylor-series

\begin{align}
	\lim\limits_{\sy \rightarrow \sx} \left(\frac{\partial G_k(\tx, \ty)}{\partial \tn(\tx) \partial \tn(\ty)} - \frac{\partial G_0(\tx, \ty)}{\partial \tn(\tx) \partial \tn(\ty)}\right) = 
	\frac{\te^{- \ti kr}}{4 \pi r^3}\left[3 \cdot (1 - \te^{\ti kr}) + 3 \ti kr - k^2 r^2) \frac{\partial r}{\partial \tn(\tx)}\frac{\partial r}{\partial \tn(\ty)} + (1 - \te^{\ti kr} + \ti kr) \tn(\tx) \cdot \tn (\ty) \right]
\end{align}

\begin{align}
	D(r) = \frac{k^2}{8 \pi r} + \frac{\ti k^3}{14 \pi} + o(r)
\end{align}



\section{The Burton-Miller formulation}

\begin{equation}
	C(\sx) p (\sx) = \int_{\Gamma}\left[-\frac{\partial{G}_k (\sx, \sy)}{\partial \sn_y} p(\sy) + G_k(\sx,\sy)\frac{\partial p(\sy)}{\partial \sn_y}\right] \td \Gamma_y + p_{in}(\sx)
\end{equation}

The normal derivative ($ \partial / \partial \sn_x$):

\begin{equation}
	C(\sx) \frac{\partial p(\sx)}{\sn_x} = \int_{\Gamma}\left[-\frac{\partial^2{G}_k (\sx, \sy)}{\partial \sn_y \partial \sn_x} p(\sy) + \frac{\partial G_k(\sx,\sy)}{\partial \sn_x}\frac{\partial p(\sy)}{\partial \sn_y}\right] \td \Gamma_y + \frac{\partial p_{in}(\sx)}{\partial \sn_x}
\end{equation}

Using the Burton-Miller formulation for surface points ($C(\sx) = 1/2$), the linear combination is calculated with the coupling constant $\alpha$

\begin{align}
	\alpha & \int_\Gamma \frac{\partial^2{G}_k (\sx, \sy)}{\partial \sn_y \partial \sn_x} p(\sy) \td \Gamma_y -
	\alpha \int_\Gamma \frac{\partial{G}_k (\sx, \sy)}{\partial \sn_x} \frac{\partial p(\sy)}{\partial \sn_y} \td \Gamma_y = \nonumber \\
	& = \int_{\Gamma}\left[-\frac{\partial{G}_k (\sx, \sy)}{\partial \sn_y} p(\sy) + G_k(\sx,\sy)\frac{\partial p(\sy)}{\partial \sn_y}\right] \td \Gamma_y - \frac{1}{2} p(\sx) - \frac{1}{2}\alpha \frac{ \partial p(\sx)}{\partial \sn_x} + p_{in}(\sx) + \alpha \frac{\partial p_{in}(\sx)}{\partial \sn_x}
\end{align}

Using shape functions $N_i(\sx)$ for the pressure and its normal derivative:

\begin{align}
	p(\sx) & = \sum\limits_{i=1}^P N_i(\sx)p_i \\
	\frac{\partial p(\sx)}{\partial \sn(\sx)} & = \sum\limits_{i=1}^P \frac{\partial N_i(\sx)}{\partial \sn(\sx)}p_i
\end{align}

Rearranging:

\begin{align}
	\frac{1}{2}p(\sx_i) & - p_{in}(\sx_i) + \frac{\alpha}{2}\frac{\partial p(\sx_i)}{\partial \sn(\sx)} - \alpha \frac{\partial p_{in}(\sx_i)}{\partial \sn(\sx)} = \nonumber \\
	& \alpha \int_\Gamma \frac{\partial{G}_k (\sx_i, \sy)}{\partial \sn(\sx)} \bN(\sy)\bq \td \Gamma(\sy)
	- \alpha \int_\Gamma \frac{\partial^2{G}_k (\sx_i, \sy)}{\partial \sn_y \partial \sn_x} \bN(\sy) \bp \td \Gamma(\sy) - \nonumber \\
	& \int_{\Gamma}\left[\frac{\partial{G}_k (\sx_i, \sy)}{\partial \sn_y} \bN(\sy) \bp - G_k(\sx_i,\sy) \bN(\sy) \bq \right] \td \Gamma (\sy) 
\end{align}

In matrix form:

\begin{align}
	\frac{1}{2} \mathbf{p} - \mathbf{p}_{in} + \frac{\alpha}{2} \mathbf{q} - \alpha \mathbf{q}_{in} = \bH \bp - \bG \bq
\end{align}

With

\begin{align}
	\bH_{ij} & = \int_{\Gamma}\frac{\partial{G}_k (\sx_i, \sy)}{\partial \sn_y} N_j(\sy) \td \Gamma(\sy)
	- \alpha \int_\Gamma \frac{\partial^2{G}_k (\sx_i, \sy)}{\partial \sn_y \partial \sn_x} N_j(\sy) \td \Gamma(\sy) \nonumber \\
	\bG_{ij} & = \int_{\Gamma} G_k(\sx_i,\sy) N_j(\sy) \td \Gamma(\sy) - \alpha \int_\Gamma \frac{\partial{G}_k (\sx_i, \sy)}{\partial \sn(\sx)} N_j(\sy) \td \Gamma(\sy)
\end{align}




\section{Treatment of hypersingular integrals}

Using the Burton-Miller formulation for surface points ($C(\sx) = 1/2$), the linear combination is calculated with the coupling constant $\alpha$

\begin{align}
	\alpha & \int_\Gamma \frac{\partial^2{G}_k (\sx, \sy)}{\partial \sn_y \partial \sn_x} p(\sy) \td \Gamma_y -
	\alpha \int_\Gamma \frac{\partial{G}_k (\sx, \sy)}{\partial \sn_x} \frac{\partial p(\sy)}{\partial \sn_y} \td \Gamma_y = \nonumber \\
	& = \int_{\Gamma}\left[-\frac{\partial{G}_k (\sx, \sy)}{\partial \sn_y} p(\sy) + G_k(\sx,\sy)\frac{\partial p(\sy)}{\partial \sn_y}\right] \td \Gamma_y - \frac{1}{2} p(\sx) - \frac{1}{2}\alpha \frac{ \partial p(\sx)}{\partial \sn_x} + p_{in}(\sx) + \alpha \frac{\partial p_{in}(\sx)}{\partial \sn_x}
\end{align}

\begin{align}
	\int_\Gamma \frac{\partial G_0(\sx, \sy)}{\partial \sn_y} \td \Gamma_y & = 0 \\
	\int_\Gamma \frac{\partial^2 G_0(\sx, \sy)}{\partial \sn_x \partial \sn_y} \td \Gamma_y & = 0 \\
	\int_\Gamma \frac{\partial^2 G_0(\sx, \sy)}{\partial \sn_x \partial \sn_y}(\sy - \sx) \td \Gamma_y & = \int_\Gamma \frac{\partial G_0(\sx, \sy)}{\partial \sn_x} \sn_y \td \Gamma_y 
\end{align}




\end{document}