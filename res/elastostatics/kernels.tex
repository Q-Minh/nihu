\documentclass{article}
\usepackage{a4wide}
\usepackage{amsmath}
\newcommand{\atanh}{\mathrm{atanh}}
\newcommand{\td}{\mathrm{d}}

\title{Analytical integrals of the elastostatics kernels}

\begin{document}

\maketitle

\section{The PDE}

\section{The displacement kernel}

The displacement kernel is defined as:
%
\begin{equation}
U_{ij}(r) = \frac{(3-4\nu)\delta_{ij} + r,_i r,_j}{16\pi\mu(1-\nu)r}
\end{equation}
%
Integration over a plane triangle is considered using the following local coordinate system:
The $\zeta$ axis is perpendicular to the element. The $\xi$ and $\eta$ axes are on the element plane, and the the $\xi$ axis is perpendicular to the opposite side, while the $\eta$ axis points towards $x_2-x_1$. Polar coordinates are introduced in the intrinsic domain. The collocation point $x_0$ is at the origin of the system.

With unit weighting function  results in expression:
%
\begin{align}
I &= \lim_{\epsilon\to 0} \int \int_{\epsilon}^{\hat{\rho}/\cos(\theta)} U(\rho,\theta) \rho \td \rho \td \theta \nonumber \\
&= \frac{\hat{\rho}}{16\pi\mu(1-\nu)}
\begin{bmatrix}
\sin\theta + (3-4\nu) q &  &  \text{symm} \\ 
- \left(\cos\!\left(\theta\right) + 1\right) &  4 (1-\nu) q - \sin\theta & \\
0 & 0 & \left(3-4\nu\right)q 
\end{bmatrix}
\end{align}
%
where
%
\begin{equation}
q = 2 \atanh\left(\tan\frac{\theta}{2}\right)
\end{equation}

\section{The traction kernel}

The traction kernel:
%
\begin{equation}
T_{ij}(r,n) = \frac{-r,_n ((1-2\nu)\delta_{ij} + 3r,_i r,_j) + (1-2\nu) (r,_i n_j - r_,j n_i)}{8\pi(1-\nu)r^2}
\end{equation}
%
In local coordinates on a plane triangle:
%
\begin{equation}
T(\rho,\theta) = 
\frac{1-2\nu}{8\pi(1-\nu)} \frac{1}{\rho^2} \begin{bmatrix}
0 & 0 & \cos\theta \\
0 & 0 & \sin\theta \\
-\cos\theta & -\sin\theta & 0
\end{bmatrix}
\end{equation}
%
and its integral is
%
\begin{align}
\lim_{\epsilon \to 0}\int \int_{\epsilon}^{\hat{\rho}/\cos\theta} T(\rho,\theta) \rho \td \rho \td \theta
&= 
\frac{1-2\nu}{8\pi(1-\nu)}
\lim_{\epsilon\to0}
\int
\left( \log \frac{\hat{\rho}}{\cos\theta} - \log \epsilon \right) \begin{bmatrix}
0 & 0 & \cos\theta \\
0 & 0 & \sin\theta \\
-\cos\theta & -\sin\theta & 0
\end{bmatrix}
\td\theta \nonumber \\
&= 
\frac{1-2\nu}{8\pi(1-\nu)}
\int
\log \frac{\hat{\rho}}{\cos\theta} \begin{bmatrix}
0 & 0 & \cos\theta \\
0 & 0 & \sin\theta \\
-\cos\theta & -\sin\theta & 0
\end{bmatrix}
\td\theta \nonumber \\
&= 
\frac{1-2\nu}{8\pi(1-\nu)}
\begin{bmatrix}
0 & 0 & a \\
0 & 0 & b \\
-a & -b & 0
\end{bmatrix}
\end{align}
%
where
%
\begin{align}
a &= \int \log \frac{\hat{\rho}}{\cos\theta} \cos\theta \td \theta
= \sin\theta \left(  \log \frac{\hat{\rho}}{\cos\theta} + 1\right) + \log \frac{\cos\frac{\theta}{2}-\sin\frac{\theta}{2}}{\cos\frac{\theta}{2}+\sin\frac{\theta}{2}}\\
b &= \int \log \frac{\hat{\rho}}{\cos\theta} \sin\theta \td \theta
=  -\cos\theta \left(  \log \frac{\hat{\rho}}{\cos\theta} + 1\right)
\end{align}

\end{document}

