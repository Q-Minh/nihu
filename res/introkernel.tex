\documentclass{article}

\usepackage{amsmath}
\usepackage{listings}
\usepackage{a4wide}
\usepackage{xcolor}
\usepackage[T1]{fontenc}
\usepackage[utf8]{inputenc}

\newcommand{\td}{\mathrm{d}}

\lstset{ %
  basicstyle=\footnotesize,        % the size of the fonts that are used for the code
  breaklines=true,                 % sets automatic line breaking
  language=C++,                 % the language of the code
  numbers=left,                    % where to put the line-numbers; possible values are (none, left, right)
  numbersep=5pt,                   % how far the line-numbers are from the code
  numberstyle=\tiny\color{gray}, % the style that is used for the line-numbers
  stepnumber=1,                    % the step between two line-numbers. If it's 1, each line will be numbered
  tabsize=2,                       % sets default tabsize to 2 spaces
}

\begin{document}

In this section we demonstrate how a new family of problems can be incorporated into the NiHu framework.
The addressed problem is three-dimensional linear isotropic elastostatics governed by the second order PDE:
%
\begin{equation}
\mu u_{i,jj} + \left(\mu + \lambda\right) u_{j,ij} = 0, x \in \Omega
\qquad t_i = \bar{t}_i, x \in \Gamma_t,
\qquad u_i = \bar{u}_i, x \in \Gamma_u
\end{equation}
%
where $u$ denotes the displacement vector field, $t$ denotes the traction vector field and $\mu$ and $\lambda$ are the Lamé-coefficients.

The corresponding BIE is formulated as
%
\begin{equation}
\int_{\Gamma} T_{ij}(x,y) u_j(y) \td \Gamma(y) - \int_{\Gamma} U_{ij}(x,y) t_{j}(y) \td \Gamma = c(x) u_i(x)
\end{equation}
%
where the displacement and traction fundamental solutions are given as
%
\begin{equation}
U_{ij} = \frac{(3-4\nu) \delta_{ij} + r,_i r,_j}{16 \pi \mu (1-\nu) r}, \qquad
T_{ij} = \frac{-r,_n ((1-2\nu)\delta{ij} + 3 r,_i r,_j) + (1-2\nu) (r,_i n_j - r,_j n_i)}{8 \pi (1-\nu) r^2}
\end{equation}

The fundamental solutions are incorporated into NiHu in three steps:
\begin{enumerate}
	\item Formulate the kernel expressions with appropriate functors
	\item Define the kernels' properties in a traits class
	\item Derive the kernel class from kernel\_base
\end{enumerate}
%
For the case of strongly singular and hypersingular kernels, like the traction kernel above, additional analytical singular integral expressions may also be needed.
The following code segments demonstrate how the strongly singular traction kernel is implemented for use in collocational formalism with general higher order curved elements.

\subsection{Kernel inputs, kernel data, kernel output and kernel expressions}

The fundamental solutions are bivariate functions of two kernel inputs.
For the case of the traction kernel, the test input is location and the trial input is location with normal vector.
The kernels are parametrised by material properties such as the shear modulus $\mu$ and Poisson's ratio $\nu$.

First, we need to define the kernel data class that stores the Poisson's ratio.
The defined class is a simple wrapper for a double data.
The shear modulus could also be incorporated into the data class, but as this parameter serves only as a scaling factor, it is omitted in the current implementation.

\begin{lstlisting}
class poisson_ratio_data {
public:
	poisson_ratio_data(double nu) :	m_nu(nu) {}
	double get_poisson_ratio(void) const { return m_nu; }
private:
	double m_nu;
};
\end{lstlisting}

Next, we provide a functor, called Tkernel, that evaluates the traction kernel. The functor needs to define its return type as well in the form of an internal typedef.

\begin{lstlisting}
struct Tkernel {
	typedef Eigen::Matrix<double, 3, 3> return_type;
	
	return_type operator()(
		location_input_3d const &x,
		location_normal_input_3d const &y,
		poisson_ratio_data const &data) {
		auto nu = data.get_poisson_ratio();
		auto rvec = y.get_x() - x.get_x();
		auto const &n = y.get_unit_normal();
		auto r = rvec.norm();
		auto dr = rvec.normalized();
		auto rdn = dr.dot(n);
		return (-rdn * ((1.-2.*nu)*return_type::Identity() + 3.*(dr*dr.transpose()))
			+ (1.-2.*nu) * (dr*n.transpose()-n*dr.transpose())
			) / (8.*M_PI*(1.-nu)*r*r);
	}
};
\end{lstlisting}

\subsection{Kernel properties}

Now we can declare the kernel class:
%
\begin{lstlisting}
class elastostatics_3d_T_kernel;
\end{lstlisting}
%
and define its properties by specialising the template kernel\_traits to the newly declared kernel:
%
\begin{lstlisting}
template <>
struct kernel_traits<elastostatics_3d_T_kernel>
{
	typedef location_input_3d test_input_t;
	typedef location_normal_input_3d trial_input_t;
	typedef collect<poisson_ratio_data> data_t;
	typedef single_brick_wall<Tkernel>::type output_t;
	
	typedef asymptotic::inverse<2> far_field_behaviour_t;
	typedef gauss_family_tag quadrature_family_t;
	typedef interval_estimator<
		typename reciprocal_distance_kernel_interval<2, GLOBAL_ACCURACY>::type
	> complexity_estimator_t;
	
	static bool const is_symmetric = false;
	static bool const is_singular = true;
};
\end{lstlisting}
%
The first four typedefs define the kernel's test and trial input types, the kernel's parameter data type and the kernel's output type.
As mentioned before, NiHu provides a mechanism to split the kernel's output expression into a series of subexpressions---called bricks---, and instantiate a complex output structure---named wall--- to allow optimised parallel kernel evaluations.
For simplicity, this feature is not exploited in this example, and the kernel's output is a single-brick-wall, computed using the functor Tkernel.

The following typedef defines the kernel's regular far field behaviour as $O(1/r^2)$.
Far field regular integrals will be computed using standard Gaussian quadratures, where the quadrature order (complexity) is defined based on the distance between kernel input locations, by an interval estimator class.

The kernel is defined as being symmetric, this information is exploited with Galerkin discretisations.
Finally, the kernel is defined as singular.

\subsection{Kernel class definition}

Finally the kernel class is defined, derived from kernel\_base using the CRTP pattern. The class defines a constructor taking the Poisson's number as argument:
%
\begin{lstlisting}
class elastostatics_3d_T_kernel :
	public kernel_base<elastostatics_3d_T_kernel>
{
public:
	elastostatics_3d_T_kernel(double nu) :
		kernel_base<elastostatics_3d_T_kernel>(poisson_ratio_data(nu))
	{
	}
};
\end{lstlisting}

\subsection{Further definitions}

The kernel has been defined as singular.
This necessitates the definition of further options in the specialisation of template singular\_kernel\_traits:
%
\begin{lstlisting}
template <>
struct singular_kernel_traits<elastostatics_3d_T_kernel>
{
	typedef asymptotic::inverse<2> singularity_type_t;
	typedef elastostatics_3d_T_kernel singular_core_t;
};
\end{lstlisting}
%
The kernel's singularity order is defined as $O(1/r^2)$.
From this information the compiler deduces that the singularity is strong, it needs to be evaluated in CPV sense, and will not be handled by a blind quadrature.
The library user has two choices here: Either specialise the class template singular\_integral\_shortcut for a specific discretisation option---e.g. define an analytical expression of the integral on a plane triangle with a constant weighting function---or specialise a unified integration method for strongly singular and hypersingular integrals, suitable for a general $O(1/r^2)$ singularity.
This example demonstrates the latter, i.e. applying Guiggiani's method, implemented in the library in a generic way.

Guiggiani's method is a singularity subtraction technique in polar coordinates.
The kernel's singular part is integrated analytically, and the remaining regular part is integrated by Gaussian quadratures.
In order to facilitate this technique, we need to define the kernel's singular part in the form of a Laurent series expansion in polar intrinsic coordinates.
As the fundamental solutions of elastodynamics share the same singularity as their static counterpart, the Laurent expansions are templated on the kernel's singular core---the kernel itself in our special case.


%
The kernel is now ready to instantiate integral operators and accurately evaluate CPV integrals even on higher order curved elements.

\end{document}
