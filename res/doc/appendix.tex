\appendix

\chapter{Nearly singular integrals}

\section{Derivation of contour integrals of the Laplace HSP kernel}
\label{sec:stokes_laplace_hsp}

Stokes theorem, in index notation, is written as follows:
%
\begin{equation}
	\int_E \Psi({\bf y}),_j n_i({\bf y}) - \Psi({\bf y}),_i n_j({\bf y}) \td S({\bf y}) = \epsilon_{ijk} \oint_C \Psi({\bf y}) \td y_k
\end{equation}
%
where the notation $,_i$ stands for $\partial/\partial y_i$, and $\epsilon_{ijk}$ is the permutation symbol.
Using vector notation, application of the permutation symbol is alternatively interpreted as
%
\begin{equation}
	\epsilon_{ijk} a_i b_j c_k = \left({\bf a} \cdot \left({\bf b} \times {\bf c}\right)\right)
\end{equation}

Integral $I_0$ is converted as follows:
%
\begin{multline}
	I_0
	= \int_E G''_{x_i y_j}({\bf x}, {\bf y}) n_i({\bf x}) n_j({\bf y}) \td S({\bf y}) \\
	= -\int_E G''_{y_i y_j}({\bf x}, {\bf y}) n_i({\bf x}) n_j({\bf y}) \td S({\bf y}) \\
	= \int_E \left[ -G,_j({\bf x}, {\bf y}) n_i({\bf x}) \right],_{i} n_j({\bf y}) \td S({\bf y}) \\
	= \int_E \left\{
		\left[G,_j({\bf x}, {\bf y}) n_i({\bf x}) \right],_{j} n_i({\bf y})
		-\left[G,_j({\bf x}, {\bf y}) n_i({\bf x}) \right],_{i} n_j({\bf y})
	\right\} \td S({\bf y})
	- \int_E \left[G,_j({\bf x}, {\bf y}) n_i({\bf x}) \right],_{j} n_i({\bf y}) \td S({\bf y}).
\end{multline}
%
Stokes' theorem can be directly applied to the first integral:
%
\begin{equation}
	I_0 = \epsilon_{ijk} \oint_C G,_j({\bf x}, {\bf y}) n_i({\bf x}) \td y_k
	- \int_E G,_{jj}({\bf x}, {\bf y}) n_i({\bf x}) n_i({\bf y}) \td S({\bf y})
\end{equation}
%
The second integral can be simplified by taking into account the definition of the Green's function: $G,_{jj} = -\delta({\bf x} - {\bf y})$
%
\begin{equation}
	I_0 = n_i({\bf x}) \epsilon_{ijk} \oint_C G,_j({\bf x}, {\bf y}) \td x_k
	+ \int_E \delta({\bf x} - {\bf y}) n_i({\bf x}) n_i({\bf y}) \td S({\bf y})
\end{equation}
%
For the case of the nearly singular integrals, the singular point is outside the integration element. As a consequence, the surface integral over $E$ vanishes.

The second integeral ${\bf I}_1$ is converted as follows:
%
\begin{multline}
	I_{1p}
	= n_i({\bf x}) \int_E -G,_{ij}({\bf x}, {\bf y}) 
	(y - x_c)_p
	n_j({\bf y})
	\td S({\bf y}) \\
	= n_i({\bf x})
	\left\{ \int_E \left[-G,_{j}({\bf x}, {\bf y}) 
	(y - x_c)_p\right]_i
	n_j({\bf y})
	\td S({\bf y})
	+
	\delta_{pi}
	\int_E G,_{j}({\bf x}, {\bf y}) 
	n_j({\bf y})
	\td S({\bf y})
	\right\}
\end{multline}
%
As the second integral is $\int G_{n_y} \td y$, which equals $-\Omega({\bf x})/4\pi$, ${\bf I}_1$ simplifies to
%
\begin{equation}
	I_{1p}
	= 
	n_i({\bf x})
	\underbrace{\int_E
	\left[-G,_{j}({\bf x}, {\bf y}) 
	(y - x_c)_p\right]_i
	n_j({\bf y})
	\td S({\bf y})}_{I'_{1p}}
	-
	n_p({\bf x})
	\frac{\Omega({\bf x})}{4\pi}
\end{equation}
%
${\bf I}'_1$ is further elaborated as follows: The term $\left[G,_{j}({\bf x}, {\bf y}) (y - x_c)_p\right]_j n_i({\bf y})$ is added and subtracted to yield
%
\begin{multline}
	I'_{1p} = 
	\int_E
		\left[G,_{j}({\bf x}, {\bf y}) (y - x_c)_p\right]_j
		n_i({\bf y})
		-
		\left[G,_{j}({\bf x}, {\bf y}) (y - x_c)_p\right]_i
		n_j({\bf y})
	\td S({\bf y}) \\
	-
	\int_E
	\left[G,_{j}({\bf x}, {\bf y}) (y - x_c)_p\right]_j
	n_i({\bf y})
	\td S({\bf y})
\end{multline}
%
The first integral is simplified using Stokes' theorem. The second integral is simplified using the definition of the Green's function
%
\begin{equation}
	I'_{1p} = 
	\epsilon_{ijk}
	\oint_C
	G,_{j}({\bf x}, {\bf y}) (y - x_c)_p
	\td y_k
	-
	\delta_{pj}
	\int_E
		G,_{j}({\bf x}, {\bf y})
		n_i({\bf y})
	\td S({\bf y})
\end{equation}
%
Finally, the last integral is further simplified by subtracting and adding $G,_{i}({\bf x}, {\bf y}) n_j({\bf y})$, and applying Stokes' theorem:
%
\begin{equation}
	I'_{1p} = 
	\epsilon_{ijk}
	\oint_C
	G,_{j}({\bf x}, {\bf y}) (y - x_c)_p
	\td y_k
	-
	\epsilon_{ipk}
	\oint_C
		G({\bf x}, {\bf y})
	\td y_k
	-
	\int_E
		G,_{i}({\bf x}, {\bf y}) n_p({\bf y})
	\td S({\bf y})
\end{equation}
%
Combinig the results, the final integral is written as
%
\begin{multline}
	I_{1p} = 
	n_i({\bf x})
	\epsilon_{ijk}
	\oint_C
	G,_{j}({\bf x}, {\bf y}) (y - x_c)_p
	\td y_k
	-
	n_i({\bf x})
	\epsilon_{ipk}
	\oint_C
		G({\bf x}, {\bf y})
	\td y_k \\
	-
	n_i({\bf x})
	\int_E
		G,_{i}({\bf x}, {\bf y}) n_p({\bf y})
	\td S({\bf y})
	-
	n_p({\bf x}) \frac{\Omega({\bf x})}{4\pi}
\end{multline}
%
or, using vector notation:
%
\begin{multline}
	{\nabla N}({\bf x}_c) \cdot {\bf I}_{1} = 
	\oint_C
	\left( \nabla N({\bf x}_c) \cdot ({\bf y} - {\bf x}_c) \right)
	\left( {\bf n}({\bf x}) , \nabla G({\bf x}, {\bf y}) , \td {\bf y} \right)
	-
	\oint_C
	G({\bf x}, {\bf y})
	\left(
	{\bf n}({\bf x}),
	\nabla N({\bf x}_c),
	\td {\bf y}
	\right)
	\\
	+
	\int_E
	G'_{n_x}({\bf x}, {\bf y})
	\left(\nabla N({\bf x}_c) \cdot {\bf n}({\bf y})\right)
	\td S({\bf y})
	-
	\left( \nabla N({\bf x}_c) {\bf n}({\bf x}) \right) \frac{\Omega({\bf x})}{4\pi}
\end{multline}
