\chapter{Singular integrals}

\section{2D Laplace kernel}

\subsection{Collocation}

\subsubsection{SLP kernel}

\begin{itemize}
\item This method is implemented in function \code{laplace_2d_SLP_collocation_general}.
\item For the special case of line elements, the simplified version is implemented in function \code{laplace_2d_SLP_collocation_straight_line_second_order}.
\end{itemize}

The integral is written in the physical domain as
%
\begin{equation}
	I = \int_S G(r) N({\bf y}) \td S
\end{equation}
%
or, transformed into the reference domain with the origin being the collocation point
%
\begin{equation}
	I = \int_a^b G(r(\rho)) N(\rho) J(\rho) \td \rho
\end{equation}
%
where $\rho$ denotes the signed distance from the singular point in the reference domain, and $J$ denotes the Jacobian of the coordinate transform.

The Taylor expansion of the distance vector ${\bf r}$ with repsect to the reference distance $\rho$ is written as
%
\begin{equation}
	{\bf r} = \rho {\bf r}_1 + \frac{\rho^2}{2} {\bf r}_2 + O(\rho^3)
\end{equation}
%
As a consequence, the Taylor series expansion of the scalar squared distance $r^2$ is expressed as:
%
\begin{equation}
	r^2 = \rho^2 r_1^2 (1 + \rho A + O(\rho^2))
\end{equation}
%
where
%
\begin{equation}
	A = \frac{{\bf r}_1 {\bf r}_2}{r_1^2}.
\end{equation}
%
and the Taylor series expression of the scalar distance $r$ becomes
%
\begin{equation}
	r = |\rho| r_1 \left(1 + \frac{\rho A}{2} + O(\rho^2)\right)
\end{equation}

Using the above expansions, the Taylor expansion of the Green's function $G(r)$ around the singular point in the reference domain is written as
%
\begin{equation}
	G(r) = -\frac{1}{2\pi} \left(
		\log (|\rho| r_1) + \rho \frac{A}{2} + O(\rho^2)
		\right)
\end{equation}


The Taylor series expansion of the shape function and the Jacobian over the element are expressed as
%
\begin{align}
	N(\rho) &= N_0 + N_1 \rho + N_2 \rho^2 \\
	J(\rho) &= J_0 + J_1 \rho
\end{align}
%
and the Taylor expansion of their product $N(\rho) J(\rho)$ is
%
\begin{equation}
	N(\rho) J(\rho) = C_0 + C_1 \rho + C_2 \rho^2
\end{equation}
%
where
%
\begin{align}
	C_0 &= N_0 J_0 \\
	C_1 &= N_0 J_1 + N_1 J_0 \\
	C_2 &= N_2 J_0 + N_1 J_1
\end{align}
%
Substituting the Taylor series expansions into the total integral in the reference domain results in a singular line integral
%
\begin{equation}
	I = -\frac{1}{2\pi} \int_{a}^{b}
	 \left(
		\log (|\rho| r_1) + \rho \frac{A}{2} + O(\rho^2)
	\right)
	\left( C_0 + C_1 \rho + C_2 \rho^2 \right) \td\rho
\end{equation}

The singular part of the integrand is
%
\begin{equation}
	F_0 = -\frac{1}{2\pi} \log \left(|\rho| r_1\right) \left( C_0 + C_1 \rho + C_2 \rho^2 \right)
\end{equation}
%
and its integral over the element can be computed analytically, using the expression
%
\begin{equation}
	\int \log(r_1 |\rho|) \rho^n \td \rho = \frac{\rho^{n+1}}{n+1} \left( \log\left(r_1 |\rho|\right) - \frac{1}{n+1} \right)
\end{equation}
The remaining regular part of the integral is computed numerically, using standard Gaussian quadrature.


\subsubsection{DLP kernel over a general curved element}

The kernel is expressed as
%
\begin{equation}
	G'_{n_y} = \frac{-1}{2\pi r} r'_{n_y}
	= \frac{-1}{2\pi r^2} \left( {\bf r} {\bf n}_y \right)
\end{equation}
%
As the kernel is integrated over the element, a more convenient expression is
%
\begin{equation}
	G'_{n_y} J = \frac{-1}{2\pi r^2} \left( {\bf r} {\bf J} \right)
\end{equation}

Substituting the Taylor series expressions for $1/r^2$, ${\bf r}$ and ${\bf J}$, the integrand is expressed as
%
\begin{equation}
	G'_{n_y} J = \frac{-1}{2\pi} \left(\frac{1-A\rho + O(\rho^2)}{\rho^2 r_1^2}\right) 
	\left( {\bf r}_1 \rho + {\bf r}_2 \frac{\rho^2}{2} + O(\rho^3) \right) 
	\left( {\bf J}_0 + \rho {\bf J}_1 + O(\rho^2) \right)
\end{equation}
%
Taking into account that ${\bf r}_1$ is perpecdicular to ${\bf J}_0$, it is observed that the integrand is regular, and can be evaluated using standard Gaussian quadrature, if it is insured that the reference location is not sampled.
This is easily done by dividing the element into two at the reference point, and integrating numerically with Gaussian quadratures over the two subelements.



\subsubsection{HSP kernel over a general curved element}

The kernel is expressed as
%
\begin{equation}
	G''_{n_x n_y} = \frac{1}{2\pi r^2} \left( 2 r'_{n_x} r'_{n_y} + {\bf n}_x {\bf n}_y \right)
	= \frac{1}{2\pi r^2} \left( {\bf n}_x {\bf n}_y - 2 \frac{{\bf r} {\bf n}_x}{r} \frac{{\bf r}{\bf n}_y}{r} \right)
\end{equation}
%
When the kernel is integrated over the element, a more convenient expression is to compute the kernel's value multiplied by the Jacobian
%
\begin{equation}
	G''_{n_x n_y} J = 
	\frac{1}{2\pi r^2} \left( {\bf n}_x {\bf J} - 2 \frac{{\bf r} {\bf n}_x}{r} \frac{{\bf r}{\bf J}}{r} \right)
\end{equation}

The general collocational integral for curved elements and a general shape function is written as
%
\begin{align}
	I = \int_S G''_{n_x n_y}(r) N({\bf y}) \td y
	= \int_a^b G''_{n_x n_y}(r(\rho)) N(\rho) J(\rho) \td \rho
\end{align}
%
where $\rho$ denotes the signed distance from the singular point in the reference domain.

In order evaluate the singular part of the integral analytically, the following Taylor series expansions are introduced:
\begin{align}
	{\bf r}(\rho) &= \rho {\bf r_1} + \frac{\rho^2}{2} {\bf r}_2 + O(\rho^3) \\
	{\bf d}(\rho) &= \frac{\td {\bf r}(\rho)}{\td \rho} = {\bf r_1} + \rho {\bf r}_2 + O(\rho^2)\\
	{\bf J}(\rho) &= {\bf T} {\bf d}(\rho) = {\bf J}_0 + \rho {\bf J}_1 + O(\rho^2) \\
	r^2 &= \rho^2 r_1^2 \left( 1 + A \rho + O(\rho^2) \right)
	,\quad A = \frac{{\bf r}_1 {\bf r}_2}{r_1^2} \\
	\frac{1}{r^2(\rho)} &\approx \frac{1 - A \rho + O(\rho^2)}{\rho^2 r_1^2}
\end{align}
%
In the above expressions, ${\bf T}$ denotes the matrix rotating by $-\pi/2$ in the positive direction.

The Green's function $G''$, multiplied with the Jacobian $J$ in the local frame of reference is approximated as (replacing $n_y$ by $J$)
%
\begin{equation}
	G''(\rho) J(\rho)
	= \frac{1}{2\pi r^2(\rho)} \left( {\bf n}_x {\bf J}(\rho) - 2 \frac{{\bf r}(\rho) {\bf n}_x}{r(\rho)} \frac{{\bf r}(\rho) {\bf J}(\rho)}{r(\rho)}\right)
\end{equation}
%
The second term in the bracketed part is $O(\rho^2)$, as ${\bf r}_1$ ${\bf n}_x = 0$ and ${\bf r}_1 {\bf J}_0 = 0$.
The first order Taylor series approximation of the first term is
%
\begin{equation}
	G''(\rho) J(\rho) \approx \frac{1 - A \rho}{2\pi \rho^2 r_1^2} {\bf n}_x \left({\bf J}_0 + \rho {\bf J}_1\right) 
\end{equation}
%
Exploiting that ${\bf n}_x \left({\bf J}_0 + \rho {\bf J}_1\right) = J_0 (1 + \rho A)$, and $r_1^2 = J_0^2$, the final first order Taylor series expansion of the kernel is
%
\begin{equation}
	G''(\rho) J(\rho) \approx \frac{1}{2\pi \rho^2 J_0}
\end{equation}
%
and the singular part of the integrand is
%
\begin{equation}
	G''(\rho) N(\rho) J(\rho) \approx \frac{1}{2\pi J_0} \left( \frac{N_0}{\rho^2} +  \frac{N_1}{\rho} \right)
\end{equation}
%
This integral is evaluated analytically using Hadamard Finite Part integration, and the remaining part is evaluated numerically, using Gaussian quadratures.