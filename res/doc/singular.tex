\chapter{Singular integrals}

\section{2D Laplace kernel}

\subsection{Collocation}

\subsubsection{Integration over curved elements}

In order to define the collocational integrals over curved elements in a general way, we introduce some series expansions of the geometrical properties of curved elements.

In the followings, the variable $\rho$ will denote the signed distance from the collocation point in the 1D reference domain.

The Taylor series expansion of the distance vector ${\bf r} = {\bf y} - {\bf x}_0$ with repsect to the reference distance $\rho$ is written as
%
\begin{equation}
	{\bf r} = \rho {\bf r}_1 + \frac{\rho^2}{2} {\bf r}_2 + O(\rho^3)
	\label{eq:taylor_expansion_rvec}
	,
\end{equation}
%
where ${\bf r}_1 = \left.{\bf r}'_{\rho}\right|_{\rho = 0}$ and ${\bf r}_2 = \left.{\bf r}''_{\rho}\right|_{\rho = 0}$. The vector ${\bf r}_1$ is tangetial to the element in the collocation point.

The Taylor series expansion of the scalar squared distance $r^2$ is expressed as:
%
\begin{equation}
	r^2 = {\bf r} {\bf r} = \rho^2 r_1^2 (1 + \rho A + O(\rho^2))
	\label{eq:taylor_expansion_r2}
	,
\end{equation}
%
where
%
\begin{equation}
	A = \frac{{\bf r}_1 {\bf r}_2}{r_1^2}
	.
\end{equation}
%
The Taylor series expression of the scalar distance $r$ becomes
%
\begin{equation}
	r = |\rho| r_1 \left(1 + \frac{\rho A}{2} + O(\rho^2)\right)
	.
	\label{eq:taylor_expansion_r}
\end{equation}
%
Furthermore, the series expansion of the inverse squared distance can be written as
%
\begin{equation}
	\frac{1}{r^2(\rho)} = \frac{1 - A \rho + O(\rho^2)}{\rho^2 r_1^2}
	,
	\label{eq:taylor_expansion_rmin2}
\end{equation}
%
and the Taylor series expansion of the inverse distance is
%
\begin{equation}
	\frac{1}{r(\rho)} = \frac{1 - \frac{A}{2} \rho + O(\rho^2)}{|\rho| r_1}
	,
	\label{eq:taylor_expansion_rmin1}
\end{equation}

Besides the Taylor series expansion of the distance, the expansion of the Jacobian is also of interest:
%
\begin{align}
	{\bf d}(\rho) &= \frac{\td {\bf r}(\rho)}{\td \rho} = {\bf r_1} + \rho {\bf r}_2 + O(\rho^2)\\
	{\bf J}(\rho) &= {\bf T} {\bf d}(\rho) = {\bf J}_0 + \rho {\bf J}_1 + O(\rho^2), \quad {\bf J}_0 = {\bf T} {\bf r}_1, {\bf J}_1 = {\bf T} {\bf r}_2
	.
	\label{eq:taylor_expansion_Jvec}
\end{align}
%
In the above expressions, ${\bf T}$ denotes the rotation matrix rotating by $-\pi/2$ in the positive direction.

The series expansion of the squared Jacobian is written as
%
\begin{equation}
	J^2 = J_0^2\left(1 + 2 A \rho + O(\rho^2)\right)
	,
	\label{eq:taylor_expansion_J2}
\end{equation}
%
and the series expansion of the scalar Jacobian is given as
%
\begin{equation}
	J = J_0 + \rho J_1 + O(\rho^2)
	= r_1\left(1 + \rho A  + O(\rho^2)\right)
	.
	\label{eq:taylor_expansion_J}
\end{equation}

Finally, the Taylor series expansion of the shape function over the element is expressed as
%
\begin{equation}
	N(\rho) = N_0 + N_1 \rho + N_2 \rho^2
	.
	\label{eq:taylor_expansion_N}
\end{equation}


\subsubsection{SLP kernel}

\begin{itemize}
\item This method is implemented in function \code{laplace_2d_SLP_collocation_general}.
\item For the special case of line elements, the simplified version is implemented in function \code{laplace_2d_SLP_collocation_straight_line_second_order}.
\end{itemize}

The integral is written in the physical domain as
%
\begin{equation}
	I = \int_S G(r) N({\bf y}) \td S
	,
\end{equation}
%
or, transformed into the reference domain with the origin being the collocation point,
%
\begin{equation}
	I = \int_a^b G(r(\rho)) N(\rho) J(\rho) \td \rho
	.
\end{equation}

Using \eqref{eq:taylor_expansion_r}, the Taylor expansion of the Green's function $G(r)$ around the singular point in the reference domain is written as
%
\begin{equation}
	G(r) = -\frac{1}{2\pi} \left(
		\log (|\rho| r_1) + \rho \frac{A}{2} + O(\rho^2)
		\right)
	.
\end{equation}

The Taylor series expansion of the product $N(\rho) J(\rho)$ is
%
\begin{equation}
	N(\rho) J(\rho) = C_0 + C_1 \rho + C_2 \rho^2
	,
\end{equation}
%
where
%
\begin{align}
	C_0 &= N_0 J_0, \\
	C_1 &= N_0 J_1 + N_1 J_0, \\
	C_2 &= N_2 J_0 + N_1 J_1.
\end{align}
%
Substituting the Taylor series expansions into the total integral in the reference domain results in a singular line integral
%
\begin{equation}
	I = -\frac{1}{2\pi} \int_{a}^{b}
	 \left(
		\log (|\rho| r_1) + \rho \frac{A}{2} + O(\rho^2)
	\right)
	\left( C_0 + C_1 \rho + C_2 \rho^2 \right) \td\rho
	.
\end{equation}

The singular part of the integrand is
%
\begin{equation}
	F_0 = -\frac{1}{2\pi} \log \left(|\rho| r_1\right) \left( C_0 + C_1 \rho + C_2 \rho^2 \right)
	,
\end{equation}
%
and its integral over the element can be computed analytically, using the expression
%
\begin{equation}
	\int \log(r_1 |\rho|) \rho^n \td \rho = \frac{\rho^{n+1}}{n+1} \left( \log\left(r_1 |\rho|\right) - \frac{1}{n+1} \right)
	.
\end{equation}
%
The remaining regular part of the integral is computed numerically, using standard Gaussian quadrature.


\subsubsection{DLP kernel over a general curved element}

The DLP kernel is expressed as
%
\begin{equation}
	G'_{n_y} = \frac{-1}{2\pi r} r'_{n_y}
	= \frac{-{\bf r} {\bf n}_y}{2\pi r^2} 
	.
\end{equation}
%
As the kernel is integrated over the element, a more convenient expression is
%
\begin{equation}
	G'_{n_y} J = \frac{-{\bf r} {\bf J}}{2\pi r^2} 
	.
\end{equation}

Substituting the Taylor series expressions for $1/r^2$ \eqref{eq:taylor_expansion_rmin2}, ${\bf r}$ \eqref{eq:taylor_expansion_rvec} and ${\bf J}$ \eqref{eq:taylor_expansion_Jvec}, the integrand is expressed as
%
\begin{equation}
	G'_{n_y} J = \frac{-1}{2\pi} \left(\frac{1-A\rho + O(\rho^2)}{\rho^2 r_1^2}\right) 
	\left( {\bf r}_1 \rho + {\bf r}_2 \frac{\rho^2}{2} + O(\rho^3) \right) 
	\left( {\bf J}_0 + \rho {\bf J}_1 + O(\rho^2) \right)
	.
\end{equation}
%
Taking into account that ${\bf r}_1$ is perpendicular to ${\bf J}_0$, it is observed that the integrand is regular, and can be evaluated using standard Gaussian quadrature, if it is ensured that the reference location is not sampled.
This is easily done by dividing the element into two at the reference point, and integrating numerically with Gaussian quadratures over the two subelements.


\subsubsection{HSP kernel over a general curved element}

The HSP kernel is expressed as
%
\begin{equation}
	G''_{n_x n_y} = \frac{1}{2\pi r^2} \left( 2 r'_{n_x} r'_{n_y} + {\bf n}_x {\bf n}_y \right)
	= \frac{1}{2\pi r^2} \left( {\bf n}_x {\bf n}_y - 2 \frac{{\bf r} {\bf n}_x}{r} \frac{{\bf r}{\bf n}_y}{r} \right)
	.
\end{equation}
%

The Green's function $G''$, multiplied with the Jacobian $J$ in the local frame of reference is approximated as (replacing ${\bf n}_y$ by ${\bf J}$)
%
\begin{equation}
	G''(\rho) J(\rho)
	= \frac{1}{2\pi r^2(\rho)} \left( {\bf n}_x {\bf J}(\rho) - 2 \frac{{\bf r}(\rho) {\bf n}_x}{r(\rho)} \frac{{\bf r}(\rho) {\bf J}(\rho)}{r(\rho)}\right)
	.
\end{equation}
%
The second term in the bracketed part is $O(\rho^2)$, as ${\bf r}_1$ ${\bf n}_x = 0$ and ${\bf r}_1 {\bf J}_0 = 0$.
Applying the Taylor series expression \eqref{eq:taylor_expansion_rmin2} of $1/r^2$, the first order Taylor series approximation of the first term is
%
\begin{equation}
	G''(\rho) J(\rho) \approx \frac{1 - A \rho}{2\pi \rho^2 r_1^2} {\bf n}_x \left({\bf J}_0 + \rho {\bf J}_1\right) 
	.
\end{equation}
%
Exploiting that ${\bf n}_x \left({\bf J}_0 + \rho {\bf J}_1\right) = J_0 (1 + \rho A)$, and $r_1^2 = J_0^2$, the final first order Taylor series expansion of the kernel is
%
\begin{equation}
	G''(\rho) J(\rho) \approx \frac{1}{2\pi \rho^2 J_0}
	,
\end{equation}
%
and the singular part of the integrand is
%
\begin{equation}
	G''(\rho) N(\rho) J(\rho) \approx \frac{1}{2\pi J_0} \left( \frac{N_0}{\rho^2} +  \frac{N_1}{\rho} \right)
	.
\end{equation}
%
This integral is evaluated analytically using Hadamard Finite Part integration, and the remaining part is evaluated numerically, using Gaussian quadratures.

The Hadamard Finite Part integration is performed as follows:
%
\begin{align}
	\int_{-\rho_1}^{\rho_2} \left( \frac{N_0}{\rho^2} +  \frac{N_1}{\rho} \right) \td \rho
	&= \int_{-\rho_1}^{-\epsilon} \left( \frac{N_0}{\rho^2} +  \frac{N_1}{\rho} \right) \td \rho
	+ \int_{\epsilon}^{\rho_2} \left( \frac{N_0}{\rho^2} +  \frac{N_1}{\rho} \right) \td \rho \\
	&= \left[ -\frac{N_0}{\rho} +  N_1\log(|\rho|) \right]_{-\rho_1}^{-\epsilon} +
	\left[ -\frac{N_0}{\rho} +  N_1\log(|\rho|) \right]_{\epsilon}^{\rho_2} \\
	&=  \cancel{\frac{2 N_0}{\epsilon} }
	- \left( \frac{N_0}{\rho_1} + \frac{N_0}{\rho_2} \right)
	+  N_1\log \frac{\rho_2}{\rho_1} 
\end{align}
