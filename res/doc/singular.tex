\chapter{Singular integrals}

This chapter describes the specialized singular integrals implemented in \NiHu{}.
The described integrals are included into the system via the shortcut class \code{singular_integral_shortcut}.

\section{2D Laplace kernel}

\subsection{Collocation}

\subsubsection{Integration over curved elements}

In order to define the collocational singular integrals over curved elements in a general way, we introduce series expansions of the geometrical properties of curved elements around the collocational point ${\bf x}_0$.

In the followings, the variable $\rho = \eta - \xi_0$ will denote the signed distance from the collocation point in the 1D reference domain $[-1, +1]$.

The Taylor series expansion of the distance vector ${\bf r} = {\bf y} - {\bf x}_0$ with repsect to the reference distance $\rho$ is written as
%
\begin{equation}
	{\bf r} = \rho {\bf r}_1 + \frac{\rho^2}{2} {\bf r}_2 + O(\rho^3)
	\label{eq:taylor_expansion_rvec}
	,
\end{equation}
%
where ${\bf r}_1 = \left.{\bf r}'_{\rho}\right|_{\rho = 0}$ and ${\bf r}_2 = \left.{\bf r}''_{\rho}\right|_{\rho = 0}$. 
Vector ${\bf r}_1$ is tangetial to the element in the collocation point, and is perpendicular to the element normal.
Using \eqref{eq:taylor_expansion_rvec}, The Taylor series expansion of the scalar squared distance $r^2$ is expressed as:
%
\begin{equation}
	r^2 = {\bf r} {\bf r} = \rho^2 r_1^2 (1 + \rho A + O(\rho^2))
	\label{eq:taylor_expansion_r2}
	,
\end{equation}
%
where
%
\begin{equation}
	A = \frac{{\bf r}_1 {\bf r}_2}{r_1^2}
	.
\end{equation}
%
By taking the square root of \eqref{eq:taylor_expansion_r2}, The Taylor series expression of the scalar distance $r$ becomes
%
\begin{equation}
	r = |\rho| r_1 \left(1 + \frac{\rho A}{2} + O(\rho^2)\right)
	.
	\label{eq:taylor_expansion_r}
\end{equation}
%
Furthermore, the series expansion of the inverse squared distance can be written as
%
\begin{equation}
	\frac{1}{r^2(\rho)} = \frac{1 - A \rho + O(\rho^2)}{\rho^2 r_1^2}
	,
	\label{eq:taylor_expansion_rmin2}
\end{equation}
%
and the Taylor series expansion of the inverse distance is
%
\begin{equation}
	\frac{1}{r(\rho)} = \frac{1 - \frac{A}{2} \rho + O(\rho^2)}{|\rho| r_1}
	,
	\label{eq:taylor_expansion_rmin1}
\end{equation}

Besides the Taylor series expansion of the distance, the expansion of the Jacobian is also of interest:
%
\begin{align}
	{\bf d}(\rho) &= \frac{\td {\bf r}(\rho)}{\td \rho} = {\bf r_1} + \rho {\bf r}_2 + O(\rho^2)\\
	{\bf J}(\rho) &= {\bf T} {\bf d}(\rho) = {\bf J}_0 + \rho {\bf J}_1 + O(\rho^2), \quad {\bf J}_0 = {\bf T} {\bf r}_1, {\bf J}_1 = {\bf T} {\bf r}_2
	.
	\label{eq:taylor_expansion_Jvec}
\end{align}
%
In the above expressions, ${\bf T}$ denotes the 2D rotation matrix rotating by $-\pi/2$ in the positive direction.
From \eqref{eq:taylor_expansion_Jvec}, the series expansion of the squared Jacobian is written as
%
\begin{equation}
	J^2 = J_0^2\left(1 + 2 A \rho + O(\rho^2)\right)
	,
	\label{eq:taylor_expansion_J2}
\end{equation}
%
and the series expansion of the scalar Jacobian simplifies to
%
\begin{equation}
	J = J_0 + \rho J_1 + O(\rho^2)
	= r_1\left(1 + \rho A  + O(\rho^2)\right)
	.
	\label{eq:taylor_expansion_J}
\end{equation}

Finally, the Taylor series expansion of the shape function over the element is expressed as
%
\begin{equation}
	N(\rho) = N_0 + N_1 \rho + N_2 \rho^2
	.
	\label{eq:taylor_expansion_N}
\end{equation}
%
In \NiHu{}, shape functions are at most second order polynomials, so \eqref{eq:taylor_expansion_N} is exact.


\subsubsection{SLP kernel}

\paragraph{The general case}

The case described here is implemented in function \code{laplace_2d_SLP_collocation_curved}.

The integral is written in the physical domain as
%
\begin{equation}
	I = \int_S G(r) N({\bf y}) \td S, \qquad G(r) = \frac{-\log(r)}{2\pi}
	,
\end{equation}
%
or, transformed into the reference domain with the origin being the collocation point,
%
\begin{equation}
	I = \int_a^b G(r(\rho)) N(\rho) J(\rho) \td \rho
	.
\end{equation}

Using \eqref{eq:taylor_expansion_r}, the Taylor expansion of the Green's function $G(r)$ around the singular point in the reference domain is written as
%
\begin{equation}
	G(r) = -\frac{1}{2\pi} \left(
		\log (|\rho| r_1) + \rho \frac{A}{2} + O(\rho^2)
		\right)
	,
\end{equation}
%
where use has been made of the Taylor expansion $\log(1+x) \approx x$ around $x=0$.

The Taylor series expansion of the product $N(\rho) J(\rho)$ is
%
\begin{equation}
	N(\rho) J(\rho) = C_0 + C_1 \rho + C_2 \rho^2
	,
\end{equation}
%
where
%
\begin{align}
	C_0 &= N_0 J_0, \\
	C_1 &= N_0 J_1 + N_1 J_0, \\
	C_2 &= N_2 J_0 + N_1 J_1.
\end{align}
%
Substituting the Taylor series expansions into the total integral in the reference domain results in a singular line integral
%
\begin{equation}
	I = -\frac{1}{2\pi} \int_{a}^{b}
	 \left(
		\log (|\rho| r_1) + \rho \frac{A}{2} + O(\rho^2)
	\right)
	\left( C_0 + C_1 \rho + C_2 \rho^2 \right) \td\rho
	.
\end{equation}

The singular part of the integrand is
%
\begin{equation}
	F_0 = -\frac{1}{2\pi} \log \left(|\rho| r_1\right) \left( C_0 + C_1 \rho + C_2 \rho^2 \right)
	,
\end{equation}
%
and its integral over the element can be computed analytically, using the expression
%
\begin{equation}
	\int \log(r_1 |\rho|) \rho^n \td \rho = \frac{\rho^{n+1}}{n+1} \left( \log\left(r_1 |\rho|\right) - \frac{1}{n+1} \right)
	, \qquad n \ge 0
	.
\end{equation}
%
The remaining regular part of the integral is computed numerically, using standard Gaussian quadrature.


\paragraph{Straight line elements}

For the special case of straight line elements, the integral can be simplified to
%
\begin{equation}
	I = \frac{-1}{2\pi} \int_{-d_1}^{d_2} G(|r|) N(r) \td r
\end{equation}
%
where $r = y-x$ is the signed distance from the collocation point.
The integral is written as
%
\begin{align}
	I 
	&= \frac{-1}{2\pi} \left(
		\int_{-d_1}^{-\epsilon} \log(-r) \left(N_0 + N_1 r + N_2 r^2 \right) \td r
		+
		\int_{\epsilon}^{d_2} \log(r) \left(N_0 + N_1 r + N_2 r^2 \right) \td r
	\right) \\
	&= \frac{-1}{2\pi} \left(
		\int_{\epsilon}^{d_1} \log(r) \left(N_0 - N_1 r + N_2 r^2 \right) \td r
		+
		\int_{\epsilon}^{d_2} \log(r) \left(N_0 + N_1 r + N_2 r^2 \right) \td r
	\right) \\
	&= 
	\frac{-1}{2\pi} \left(
		N_0 \left(F_0(d_2) + F_0(d_1)\right) +
		N_1 \left(F_1(d_2) - F_1(d_1)\right) +
		N_2 \left(F_2(d_2) + F_2(d_2)\right)
	\right)
\end{align}
%
where
%
\begin{equation}
	F_n(x) = \int \log(x) x^n \td x
	= \frac{x^{n+1}}{n+1} \left(\log x - \frac{1}{n+1}\right)
\end{equation}


simplified version is implemented in function \code{laplace_2d_SLP_collocation_straight}.


\begin{align}
\int_{S} G({\bf y}, {\bf x}) \td y
& = \frac{-1}{2\pi}\lim_{\epsilon \to 0}
\left( \int_{-d_1}^{-\epsilon} \ln |y| \td y + \int_{\epsilon}^{d_2}  \ln |y| \td y \right) \nonumber \\
& = \frac{-1}{2\pi}\lim_{\epsilon \to 0}
\left( \int_{\epsilon}^{d_1} \ln y \td y + \int_{\epsilon}^{d_2}  \ln y \td y \right) \nonumber \\
&=
\frac{d_1(1-\ln d_1) + d_2(1-\ln d_2)}{2\pi}
\end{align}



\subsubsection{DLP kernel}

\paragraph{The general case}

The DLP kernel is expressed as
%
\begin{equation}
	G'_{n_y} = \frac{-1}{2\pi r} r'_{n_y}
	= \frac{-{\bf r} {\bf n}_y}{2\pi r^2} 
	.
\end{equation}
%
As the kernel is integrated over the element, a more convenient expression is
%
\begin{equation}
	G'_{n_y} J = \frac{-{\bf r} {\bf J}}{2\pi r^2} 
	.
\end{equation}

Substituting the Taylor series expressions for $1/r^2$ \eqref{eq:taylor_expansion_rmin2}, ${\bf r}$ \eqref{eq:taylor_expansion_rvec} and ${\bf J}$ \eqref{eq:taylor_expansion_Jvec}, the integrand is expressed as
%
\begin{equation}
	G'_{n_y} J = \frac{-1}{2\pi} \left(\frac{1-A\rho + O(\rho^2)}{\rho^2 r_1^2}\right) 
	\left( {\bf r}_1 \rho + {\bf r}_2 \frac{\rho^2}{2} + O(\rho^3) \right) 
	\left( {\bf J}_0 + \rho {\bf J}_1 + O(\rho^2) \right)
	.
\end{equation}
%
Taking into account that ${\bf r}_1$ is perpendicular to ${\bf J}_0$, it is observed that the integrand is regular, and can be evaluated using standard Gaussian quadrature, if it is ensured that the reference location is not sampled.
This is easily done by dividing the element into two at the reference point, and integrating numerically with Gaussian quadratures over the two subelements.


\paragraph{Straight line elements}

\begin{equation}
\int_{S} G'_{n_y}({\bf y}, {\bf x}) \td y = 0
\end{equation}
%
as the element normal is perpendicular to the distance ${\bf y}-{\bf x}$.



\subsubsection{HSP kernel}

\paragraph{The general case}

The HSP kernel is expressed as
%
\begin{equation}
	G''_{n_x n_y} = \frac{1}{2\pi r^2} \left( 2 r'_{n_x} r'_{n_y} + {\bf n}_x {\bf n}_y \right)
	= \frac{1}{2\pi r^2} \left( {\bf n}_x {\bf n}_y - 2 \frac{{\bf r} {\bf n}_x}{r} \frac{{\bf r}{\bf n}_y}{r} \right)
	.
\end{equation}
%

The Green's function $G''$, multiplied with the Jacobian $J$ in the local frame of reference is approximated as (replacing ${\bf n}_y$ by ${\bf J}$)
%
\begin{equation}
	G''(\rho) J(\rho)
	= \frac{1}{2\pi r^2(\rho)} \left( {\bf n}_x {\bf J}(\rho) - 2 \frac{{\bf r}(\rho) {\bf n}_x}{r(\rho)} \frac{{\bf r}(\rho) {\bf J}(\rho)}{r(\rho)}\right)
	.
\end{equation}
%
The second term in the bracketed part is $O(\rho^2)$, as ${\bf r}_1$ ${\bf n}_x = 0$ and ${\bf r}_1 {\bf J}_0 = 0$.
Applying the Taylor series expression \eqref{eq:taylor_expansion_rmin2} of $1/r^2$, the first order Taylor series approximation of the first term is
%
\begin{equation}
	G''(\rho) J(\rho) \approx \frac{1 - A \rho}{2\pi \rho^2 r_1^2} {\bf n}_x \left({\bf J}_0 + \rho {\bf J}_1\right) 
	.
\end{equation}
%
Exploiting that ${\bf n}_x \left({\bf J}_0 + \rho {\bf J}_1\right) = J_0 (1 + \rho A)$, and $r_1^2 = J_0^2$, the final first order Taylor series expansion of the kernel is
%
\begin{equation}
	G''(\rho) J(\rho) \approx \frac{1}{2\pi \rho^2 J_0}
	,
\end{equation}
%
and the singular part of the integrand is
%
\begin{equation}
	G''(\rho) N(\rho) J(\rho) \approx \frac{1}{2\pi J_0} \left( \frac{N_0}{\rho^2} +  \frac{N_1}{\rho} \right)
	.
\end{equation}
%
This integral is evaluated analytically using Hadamard Finite Part integration, and the remaining part is evaluated numerically, using Gaussian quadratures.

The Hadamard Finite Part integration is performed as follows:
%
\begin{align}
	\int_{-\rho_1}^{\rho_2} \left( \frac{N_0}{\rho^2} +  \frac{N_1}{\rho} \right) \td \rho
	&= \int_{-\rho_1}^{-\epsilon} \left( \frac{N_0}{\rho^2} +  \frac{N_1}{\rho} \right) \td \rho
	+ \int_{\epsilon}^{\rho_2} \left( \frac{N_0}{\rho^2} +  \frac{N_1}{\rho} \right) \td \rho \\
	&= \left[ -\frac{N_0}{\rho} +  N_1\log(|\rho|) \right]_{-\rho_1}^{-\epsilon} +
	\left[ -\frac{N_0}{\rho} +  N_1\log(|\rho|) \right]_{\epsilon}^{\rho_2} \\
	&=  \cancel{\frac{2 N_0}{\epsilon} }
	- \left( \frac{N_0}{\rho_1} + \frac{N_0}{\rho_2} \right)
	+  N_1\log \frac{\rho_2}{\rho_1} 
\end{align}


\paragraph{Straight line elements}

Finite part integral with general shape function $N(y)$
%
\begin{align}
	I = \int_S G''_{n_x n_y}({\bf y}, {\bf x}) N({\bf y}) \td y
	&= \frac{1}{2\pi} \int_S \frac{1}{(y-x)^2} N(y) \td y \\
	&= \frac{1}{2\pi} \int_\Sigma \frac{1}{J^2 (\eta-\xi)^2} N(\eta) J \td \eta \\
	&= \frac{1}{2\pi J} \left(
		\int_a^{\xi-\epsilon} \frac{1}{(\eta-\xi)^2} N(\eta) \td \eta
		+ \int_{\xi+\epsilon}^{b} \frac{1}{(\eta-\xi)^2} N(\eta) \td \eta
	\right)
\end{align}

The shape function is written with its Taylor series around the singular point
%
\begin{equation}
	N(\eta) = N(\xi) + N'(\xi) (\eta-\xi) + \sum_{k=2}^{\infty} \frac{N^{(k)}(\xi)}{k!}(\eta-\xi)^k
\end{equation}

Substitution yields

\begin{equation}
	I = \frac{N(\xi)}{2\pi J} \left(
		\cancel{\frac{1}{2\epsilon}} + \left[ \frac{-1}{\eta-\xi} \right]_{a}^{b} 
	\right)
	+
	\frac{N'(\xi)}{2\pi J} \log \frac{|b-\xi|}{|a-\xi|}
	+ \sum_{k=2}^{\infty} \frac{N^{(k)}(\xi)}{2\pi J k! (k-1)}
		\left[ (\eta-\xi)^{k-1} \right]_{a}^{b} 
\end{equation}




\subsection{Galerkin}

\subsubsection{Integration over curved elements}

\paragraph{Face-match integration over a single curved element}

The distance vector is defined as ${\bf r} = {\bf y} - {\bf x}$, where ${\bf x}$ and ${\bf y}$ are locations in the same element.
In intrinsic coordinates, ${\bf x} = {\bf x}(\xi)$, and ${\bf y} = {\bf y}(\eta)$, and the intrinsic variables $\xi$, $\eta$ have values from the interval $[-1, +1]$.
For the case of the face match Galerkin integration, the singularity occurs when $\xi = \eta$. 
For this reason, new variables ($u$, $v$) are introduced, where $v$ is related to the singularity:
%
\begin{align}
	\xi &= u + (\pm 1 - u) v \\
	\eta &= u + (\mp 1 - u) v,
\end{align}
%
were $u \in [-1, +1]$ and $v \in [0, 1]$.
The relations divide the unit square into two triangles. 
The upper signs corresponds to the first, and the lower signs correspond to the second triangle.
Apparently, the singularity is approached when $v \to 0$, independently from $u$.
Therefore, the Taylor series expansion of the distance with repsect to $v$ is of interest.
%
\begin{equation}
	{\bf r}(v) = v \left.{\bf r}'_v\right|_{v=0} + v^2 \frac{\left.{\bf r}''_v\right|_{v=0}}{2} + O(v^3)
\end{equation}

\begin{equation}
	\left.{\bf r}'_v\right|_{v=0} = \left[{\bf y}'_{\eta} \eta'_v - {\bf x}'_{\xi} \xi'_v\right]_{v=0}
\end{equation}
%
From relations \eqref{eq:galerkin_uvtransform} it can be seen that for the $v=0$ case, $\xi = \eta = u$.
Therefore, the derivative expression simplifies to
%
\begin{equation}
	\left.{\bf r}'_v\right|_{v=0} = \mp 2 {\bf y}'(u) 
\end{equation}
%
The second derivative is expressed similarly as
%
\begin{equation}
	\left.{\bf r}''_v\right|_{v=0}
	=
	\left[
		{\bf y}''_\eta {\eta'_v}^2 - {\bf x}''_{\xi} {\xi'_v}^2
	\right]_{v=0}
	= \pm 4u {\bf y}''(u)
\end{equation}
%
Subtituting into the Taylor series expansion, the following expression is obtained:
%
\begin{equation}
	{\bf r}(v) = \mp 2 {\bf y}'(u) v 
	\pm 2u {\bf y}''(u) v^2 
	+ O(v^3)
\end{equation}
%
As a consequence, the squared distance is approximated as
%
\begin{equation}
	r^2(u,v) = 
	4v^2 |{\bf y}'(u)|^2
	\left(
	1
	-
	2 u v A(u)
	+ O(v^2)
	\right)
\end{equation}
%
where $A(u)$ is defined as
%
\begin{equation}
	A(u) = \frac{{\bf y}'(u) {\bf y}''(u)}{|{\bf y}'(u)|^2}
\end{equation}
%
Finally, the distance is approximated as
%
\begin{equation}
	r = 2v|{\bf y}'(u)| \left(1 - u v A(u) + O(v^2) \right)
	\label{eq:galerkin_face_taylor_r}
\end{equation}

\subsubsection{SLP kernel}

The integral to be computed is written in intrinsic coordinates as
%
\begin{equation}
	I = \int_{-1}^{+1}
	\int_{-1}^{+1}
	G(r(\xi,\eta)) 
	N_{\xi}(\xi) N_{\eta}(\eta) 
	J(\xi) J(\eta)
	\td \eta
	\td \xi
\end{equation}
%
and in the $uv$ plane as
%
\begin{equation}
	I^{(i)} = \int_{-1}^{+1}
	\int_{0}^{+1}
	\underbrace{G(r(u,v)) N_{\xi}(\xi^{(i)}(u,v)) N_{\eta}(\eta^{(i)}(u,v)) 
	J(\xi^{(i)}(u,v)) J(\eta^{(i)}(u,v)) 2(1-v)}_{F^{(i)}(u,v)}
	\td v
	\td u
\end{equation}
%
where $\td\xi \td\eta = 2(1-v) \td u \td v$.
Substituting the Taylor series expansion of the distance \eqref{eq:galerkin_face_taylor_r} into $G(r)$, 
%
\begin{align}
	G(r(u,v)) 
	&= \frac{-1}{2\pi} \log\left(2v|{\bf y}'(u)| \left(1 - u v A(u) + O(v^2) \right)\right) \\
	&= \frac{-1}{2\pi} \left[ 
	\log(2v|{\bf y}'(u)| )
	+
	\log \left(1 - u v A(u) + O(v^2) \right)
	\right] \\
	&= \frac{-\log(2v|{\bf y}'(u)| )
	+
	u v A(u)
	}{2\pi} 
	 + O(v^2)
\end{align}
%
Inserting \eqref{} into \eqref{}, it is seen that the singular part of $F^{(i)}(u,v)$ is
%
\begin{equation}
	F_0(u,v) = \frac{-\log(2v J(u) )}{2\pi} 
	N_{\xi}(u) N_{\eta}(u) 
	J^2(u) 2(1-v)
\end{equation}
%
Now the inner integral w.r.t. $v$ is computed by numerically evaluating the regular part, and analytically integrating the singular part
%
\begin{equation}
	\int_{-1}^{+1} \int_{0}^{+1} F^{(i)}(u,v) \td v \td u
	= 
	\int_{-1}^{+1} \int_{0}^{+1} \left( F^{(i)}(u,v) - F_0(u,v) \right) \td v \td u
	+
	\int_{-1}^{+1} \int_{0}^{+1} F_0(u,v) \td v \td u
\end{equation}
%
where the inner integral of the last singular term simplifies to
%
\begin{equation}
	\int_{0}^{+1} F_0(u,v) \td v = \frac{1}{2\pi} \left(\frac{3}{2} - \log(2 J(u)) \right) N_{\xi}(u) N_{\eta}(u) 
	J^2(u)
\end{equation}

\subsubsection{DLP kernel}

The integral to be computed is written in intrinsic coordinates as
%
\begin{align}
	I 
	&= \int_{-1}^{+1}
	\int_{-1}^{+1}
	G'_{n_y}(r(\xi,\eta)) 
	N_{\xi}(\xi) N_{\eta}(\eta)
	J(\xi) J(\eta)
	\td \eta
	\td \xi \\
	&= \int_{-1}^{+1}
	\int_{-1}^{+1}
	\frac{-{\bf r}(\xi,\eta) {\bf J}(\eta)}{2\pi r^2(\xi,\eta)}
	N_{\xi}(\xi) N_{\eta}(\eta)
	J(\xi)
	\td \eta
	\td \xi
\end{align}

\subsection{HSP kernel}

The integral to be computed is written in intrinsic coordinates as
%
\begin{align}
	I 
	&= \int_{-1}^{+1}
	\int_{-1}^{+1}
	\frac{1}{2\pi r^2} \left(
	{\bf J}(\xi) {\bf J}(\eta) - 2 \frac{{\bf r} {\bf J}(\xi)}{r} \frac{{\bf r}{\bf J}(\eta)}{r} 
	\right)	N_{\xi}(\xi)
	N_{\eta}(\eta)
	\td \eta
	\td \xi
\end{align}
%
referring back to section \label{sec:}, the second term in the bracketed expression is $O(r^2)$, and can be integrated using regular quadratures that do not sample the singular location $r=0$.
The first term is approximated with its Taylor series
%
\begin{equation}
	{\bf J}(\xi) {\bf J}(\eta) 
	= {\bf T} {\bf x}'_{\xi} \cdot {\bf T} {\bf y}'_{\eta}
	= {\bf x}'_{\xi} \cdot {\bf y}'_{\eta}
\end{equation}
%
\begin{equation}
	{\bf x}'_{\xi}(\xi(u,v)) 
	= {\bf x}'_{\xi}(\xi(u,0)) + \left.{\bf x}''_{\xi} \xi'_v \right|_{v=0} v + O(v^2) 
	= {\bf x}'_{\xi}(u) + {\bf x}''_{\xi}(u) \left(\pm 1 - u\right) v + O(v^2) 
\end{equation}
%
and
%
\begin{equation}
	{\bf y}'_{\eta}(\eta(u,v)) 
	= {\bf y}'_{\eta}(u) + {\bf y}''_{\eta}(u) \left(\mp 1 - u\right) v + O(v^2) 
\end{equation}
%
Taking the scalar product
%
\begin{align}
	{\bf J}(\xi) {\bf J}(\eta) 
	&= \left(
	{\bf x}'_{\xi}(u) + {\bf x}''_{\xi}(u) \left(\pm 1 - u\right) v + O(v^2) 
	\right)
	\left(
	{\bf y}'_{\eta}(u) + {\bf y}''_{\eta}(u) \left(\mp 1 - u\right) v + O(v^2) 	
	\right) \\
	&= J_0^2(u) \left(1 -2uvA(u) + O(v^2) \right)
\end{align}
%
Finally, the singular part in the kernel is approximated as
%
\begin{equation}
	\frac{{\bf J}(\xi) {\bf J}(\eta)}{2\pi r^2}
	= \frac{1 + O(v^2)}{2\pi 4v^2}
\end{equation}
%
and its singular part is simply $1/2\pi 4 v^2$.

Substituting the singular part into the original integral yields
%
\begin{align}
	I^{(i)}
	&= \int_{-1}^{+1}
	\int_{\alpha}^{+1}
	\frac{	N_{\xi}(\xi^{(i)}(u,v))
	N_{\eta}(\eta^{(i)}(u,v))
	(1-v)
	}{4\pi v^2}
	\td v
	\td u \\
	&= \int_{-1}^{+1}
	\int_{\alpha}^{+1}
	\frac{F_0(u) + F_1^{(i)}(u) v + O(v^2)}{v^2}
	\td v
	\td u
\end{align}
%
where
%
\begin{align}
	F_0(u) &= \frac{N_{\xi}(u) N_{\eta}(u)}{4\pi} \\
	F_1^{(i)}(u) &= \frac{N'_{\xi}(u) (\pm 1-u) N_{\eta}(u) + N_{\xi}(u) N'_{\eta}(u) (\mp 1-u) - N_{\xi}(u) N_{\eta}(u)}{4\pi}
\end{align}


\begin{align}
	F_0^{(1)} + F_0^{(2)} &= \frac{N_{\xi}(u) N_{\eta}(u)}{2\pi} \\
	F_1^{(1)} + F_1^{(2)} &= \frac{\td}{\td u} \left( u \frac{N_{\xi}(u) N_{\eta}(u)}{2\pi}\right)
\end{align}

%
\begin{align}
	I_0 &= \int_{-1}^{+1} F_0(u) \int_{\alpha(u)}^{1} \frac{\td v}{v^2} \td u \\
	&= \int_{-1}^{+1} F_0(u) \left[  \frac{1}{\alpha(u)} - 1  \right] \td u
\end{align}
%
\begin{align}
	I^{(i)}_1 &= \int_{-1}^{+1} F^{(i)}_1(u) \int_{\alpha(u)}^{1} \frac{\td v}{v} \td u \\
	&= \int_{-1}^{+1} F^{(i)}_1(u) \left[ -\log|\alpha(u)| \right] \td u
\end{align}
