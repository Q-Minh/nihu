\chapter{Geometry}
\label{sec:geometry}

This chapter describes how the geometry and the mesh is defined in \NiHu{}.

\section{Spaces}
\label{sec:space}
\index{space}

Coordinate spaces, like $\mathbb{R}^3$, are defined in the header file \texttt{space.hpp}\index{space.hpp}, in class template \code{space}\index{space class}.
Coordinate spaces are defined by a scalar type and the number of dimensions.
These are the template parameters of the \code{space}\index{space class} class:

\begin{lstlisting}
template <class scalar, unsigned dim>
class space;
\end{lstlisting}

Class template \code{space}\index{space class} provides the following interface for spaces:

\begin{description}
	\item [\code{dimension}] is the dimension of the space
	\item [\code{scalar_t}] is the scalar type of the space
	\item [\code{location_t}] is an \code{Eigen} matrix type, defining the type of a location in the space.
\end{description}

\subsection{1D Space}

\code{space_1d<scalar = double>} is a shorthand for \code{space<scalar, 1>}.
\index{space\_1d}

\subsection{2D Space}

\code{space_2d<scalar = double>} is a shorthand for \code{space<scalar, 2>}.
\index{space\_2d}

\subsection{3D Space}

\code{space_3d<scalar = double>} is a shorthand for \code{space<scalar, 3>}.
\index{space\_3d}


\section{Domains}
\label{sec:domain}
\index{domain}

Domains are subspaces of $\mathbf{R}^d$.
Domains are used e.g. as domains of shape functions.
Domains are used to define quadratures.

Domains are defined in file \code{domain.hpp}\index{domain.hpp}, and in the NiHu library files \code{lib_domain.hpp}\index{lib\_domain.hpp} and \code{lib_domain.cpp}\index{lib\_domain.cpp}.
The domain classes are derived from class \code{domain_base}\index{domain\_base}.

Each domain class provides the following interface:

\begin{description}
	\item[\code{space_t}] the space type of the domain
	\item[\code{num_corners}] the number of corneres
	\item[\code{num_edges}] the number of edges
	\item[\code{id}] each domain is assigned a unique identifier.
	The defaul value of this identifier is \code{id = 10 * domain_t::dimension + num_corners}
	\item[\code{name}] a textual identifier
	
	\item[\code{get_center()}] returns the domain's center
	\item[\code{get_corner(idx)}] returns the domain's corner
	\item[\code{get_edge(idx)}] returns the domain's edge as a two-element array
	\item[\code{get_volume()}] returns the domain's volume
\end{description}



\subsection{Line Domain}

\subsection{Triangle Domain}

\subsection{Quadrangle Domain}

\section{Shape sets}

\section{Elements}
