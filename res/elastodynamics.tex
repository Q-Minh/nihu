\documentclass{article}
\usepackage{amsmath}
\usepackage[T1]{fontenc}
\usepackage[utf8]{inputenc}
\usepackage{a4wide}

\newcommand{\te}{\mathrm{e}}
\newcommand{\ti}{\mathrm{i}}
\newcommand{\td}{\mathrm{d}}

\begin{document}

\section{Basic equations}

Linear displacement-strain relations
%
\begin{equation}
\varepsilon_{ij} = \frac{1}{2} \left(u_{i,j} + u_{j,i}\right)
\end{equation}
%
stress-strain relations:
%
\begin{equation}
\sigma_{ij} = \lambda \delta_{ij} \varepsilon_{kk} + 2 \mu \varepsilon_{ij}
\end{equation}

stress-tracton relation
%
\begin{equation}
t_i = \sigma_{ij} n_j
\end{equation}

combination
\begin{equation}
t_i = \lambda n_i u_{k,k} + \mu \left( u_{i,j} + u_{j,i} \right) n_j
\end{equation}

From Kausel, the $j$-th derivative of the $i$-th displacement component when the unit point load acts in the $l$-th direction is
%
\begin{equation}
u_{li,j} = \frac{1}{4\pi r\mu} \left(
r_{,j} A \delta_{li}
+
B r_{,j} r_{,l} r_{,i}
+
C \left(\delta_{lj} r_{,i} + \delta_{ij} r_{,l} \right)
\right)
\end{equation}



The $i$-th traction component on surface with normal $n_i$ due to a unit point source acting in direction $l$ is
%
\begin{equation}
t_{li} = \lambda n_i u_{lk,k} + \mu \left( u_{li,j} + u_{lj,i} \right) n_j
\end{equation}

The components separately are:
%
\begin{align}
u_{lk,k} &= \frac{1}{4\pi r\mu} \left(
r_{,k} A \delta_{lk}
+
B r_{,k} r_{,l} r_{,k}
+
C \left(\delta_{lk} r_{,k} + \delta_{kk} r_{,l} \right)
\right) \nonumber \\
&= \frac{r_{,l}}{4\pi r\mu} \left( A + B + 4C \right)
\end{align}
%
using that $r_{,k} r_{,k} = 1$ and $\delta_{kk} = 3$.

\begin{align}
\left(u_{li,j} + u_{lj,i}\right) n_j
&=
\frac{1}{4\pi r\mu} \left(
r_{,j} A \delta_{li}
+
2B r_{,j} r_{,l} r_{,i}
+
C \left(\delta_{lj} r_{,i} + \delta_{ij} r_{,l} \right)
+
r_{,i} A \delta_{lj}
+
C \left(\delta_{li} r_{,j} + \delta_{ji} r_{,l} \right)
\right) n_j \nonumber \\
&=
\frac{1}{4\pi r\mu} \left(
r_{,n} A \delta_{li}
+
2B r_{,n} r_{,l} r_{,i}
+
C \left(n_l r_{,i} + n_i r_{,l} \right)
+
r_{,i} A n_l
+
C \left(\delta_{li} r_{,n} + n_i r_{,l} \right)
\right)
\end{align}

the final result is
%
\begin{align}
t_{li}
=
\frac{1}{4\pi r} 
\left\{
\frac{\lambda}{\mu}
n_i r_{,l} \left( A + B + 4C \right)
+
r_{,n}
\left[(A + C) \delta_{li}
+
2B r_{,l} r_{,i}
\right]
+
C \left(n_l r_{,i} + n_i r_{,l} \right)
+
r_{,i} A n_l
+
C n_i r_{,l}
\right\}
\end{align}



\end{document}

