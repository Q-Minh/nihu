\documentclass[a4paper,11pt]{article}

\usepackage{amsmath,bm}
\usepackage{a4wide}

\newcommand{\td}{\mathrm{d}}

\begin{document}

\section{The hypersingular boundary integral equation}

The starting point of the derivation is Green' second theorem written on two regular functions $G$ and $u$ over the surface $S$
%
\begin{equation}
\int_{S} \left( G'_{n_y}({\bf y}, {\bf x}) u({\bf y}) - G({\bf y}, {\bf x}) q({\bf y})
\right) \td S_y = 0
\end{equation}
%
If ${\bf x}$ is outside the volume enclosed by $S$, then the above integral equation can be evaluated in a conventional way.
If ${\bf x} \in S$ then the surface is slightly modified to exclude the singular point from the enclosed domain, and the integral is interpreted as limiting case:
%
\begin{equation}
\lim_{\epsilon \to 0^+}
\int_{S-e_{\epsilon}+s_{\epsilon}}
\left(
G'_{n_y}({\bf y}, {\bf x}) u({\bf y}) - G({\bf y}, {\bf x}) q({\bf y})
\right) 
\td S_y = 0
\end{equation}
%
As the singular point is excluded from the surface, the integral can be differentiated with respect to the outward normal at ${\bf x}$:
%
\begin{equation}
\lim_{\epsilon \to 0^+}
\int_{S-e_{\epsilon}+s_{\epsilon}}
\left(
G''_{n_x,n_y}({\bf y}, {\bf x}) u({\bf y}) - G'_{n_x}({\bf y}, {\bf x}) q({\bf y})
\right)
\td S_y = 0
\end{equation}
%
The integrals over the truncated surface $S-e_{\epsilon}$ and the attached boundry $s_{\epsilon}$ are split into two:
%
\begin{multline}
\lim_{\epsilon \to 0^+}
\int_{S-e_{\epsilon}}
\left(
G''_{n_x,n_y}({\bf y}, {\bf x}) u({\bf y}) - G'_{n_x}({\bf y}, {\bf x}) q({\bf y})
\right)
\td S_y
\\
+
\lim_{\epsilon \to 0^+}
\int_{s_{\epsilon}}
\left(
G''_{n_x,n_y}({\bf y}, {\bf x}) u({\bf y}) - G'_{n_x}({\bf y}, {\bf x}) q({\bf y})
\right)
\td S_y = 0
\end{multline}
%
The functions $u$ and $q$ are expanded around the singular point as
%
\begin{align}
u({\bf y}) &= u({\bf x}) + \nabla u({\bf x}) ({\bf y} - {\bf x}) + O(r^2) \nonumber \\
q({\bf y}) &= \nabla u({\bf y}) {\bf n}_y = \nabla u({\bf x}) {\bf n}_y + O(r)
\end{align}
%
We substitute the expansions into the two integrals. Furthermore, we assume that $s_{\epsilon}$ is a spherical surface of radius $\epsilon$:
%
\begin{multline}
\lim_{\epsilon \to 0^+}
\int_{S-e_{\epsilon}}
\left(
G''_{n_x,n_y}({\bf y}, {\bf x}) u({\bf y})
- G'_{n_x}({\bf y}, {\bf x}) q({\bf y})
\right)
\td S_y
\\
+
\lim_{\epsilon \to 0^+}
\int_{s_{\epsilon}}
\left(
\underbrace{G''_{n_x,n_y}({\bf y}, {\bf x})}_{O(\epsilon^{-3})} \underbrace{\left[u({\bf y}) - u({\bf x}) - \nabla u({\bf x}) ({\bf y} - {\bf x})\right]}_{O(\epsilon^2)}
- \underbrace{G'_{n_x}({\bf y}, {\bf x})}_{O(\epsilon^{-2})} \underbrace{\left[q({\bf y}) - \nabla u({\bf x}) {\bf n}_y\right]}_{O(\epsilon)}
\right)
\underbrace{\td S_y}_{O(\epsilon^2)}
\\
+
\lim_{\epsilon \to 0^+}
\int_{s_{\epsilon}}
\left(
G''_{n_x,n_y}({\bf y}, {\bf x}) \left[u({\bf x}) + \nabla u({\bf x}) ({\bf y} - {\bf x})\right]
- G'_{n_x}({\bf y}, {\bf x}) \nabla u({\bf x}) {\bf n}_y
\right)
\td S_y = 0
\end{multline}
%
cancellation:
%
\begin{multline}
\lim_{\epsilon \to 0^+}
\int_{S-e_{\epsilon}}
\left(
G''_{n_x,n_y}({\bf y}, {\bf x}) u({\bf y})
- G'_{n_x}({\bf y}, {\bf x}) q({\bf y})
\right)
\td S_y
+
u({\bf x})
\lim_{\epsilon \to 0^+}
\underbrace{
\int_{s_{\epsilon}}
G''_{n_x,n_y}({\bf y}, {\bf x}) 
\td S_y
}_{b({\bf x})/\epsilon}
\\
+
\nabla u({\bf x}) 
\underbrace{
\lim_{\epsilon \to 0^+}
\int_{s_{\epsilon}}
\left(
G''_{n_x,n_y}({\bf y}, {\bf x}) ({\bf y} - {\bf x})
- G'_{n_x}({\bf y}, {\bf x}) {\bf n}_y
\right)
\td S_y
}_{\bf c({\bf x})}
= 0
\end{multline}
%

Summarising the final result:
%
\begin{equation}
\lim_{\epsilon \to 0^+}
\left\{
\int_{S-e_{\epsilon}}
\left(
G''_{n_x,n_y}({\bf y}, {\bf x}) u({\bf y})
- G'_{n_x}({\bf y}, {\bf x}) q({\bf y})
\right)
\td S_y
+
u({\bf x})
\frac{b({\bf x})}{\epsilon}
\right\}
+
\nabla u({\bf x}) 
{\bf c}({\bf x})
= 0
\end{equation}


\section{Demonstration}


2D, constant straight line element with the singular point in the middle.

\begin{equation}
G''_{n_x, n_y} = \frac{1}{2\pi r^2} \left({\bf n}_x {\bf n}_y + 2 r'_{n_x} r'_{n_y}\right)
\end{equation}
%
on the straight lines the kernel simplifies to
%
\begin{equation}
G''_{n_x, n_y} = \frac{1}{2\pi r^2}
\end{equation}

\begin{equation}
\int_{S-e_{\epsilon}} G''_{n_x,n_y} \td S_y
=
\int_{-D}^{-\epsilon} \frac{1}{2\pi r^2} \td r + \int_{\epsilon}^{D} \frac{1}{2\pi r^2} \td r
=
\frac{1}{\pi \epsilon} - \frac{1}{\pi D}
\end{equation}
%
On the semicircle $s_{\epsilon}$ the integral simplifies to
%
\begin{align}
\int_{s_{\epsilon}} G''_{n_x,n_y} \td S_y
&=
\frac{1}{2\pi\epsilon^2}\int_{s_{\epsilon}} \left({\bf n}_x{\bf n}_y + 2 r'_{n_x} r'_{n_y}\right) \td S_y \nonumber \\
&=
\frac{1}{2\pi \epsilon^2} \int_{-\pi/2}^{\pi/2} \left(\cos\theta + 2 \cos\theta (-1)\right) \epsilon \td \theta \nonumber \\
&= \frac{-1}{\pi\epsilon}
\end{align}
%
Apparently, the two unbounded terms in the two parts of the integrals cancel each other, yielding the finite result $-2/D$.


\begin{equation}
\lim_{\epsilon\to0^+}\int_{S-S_{\epsilon}} G({\bf y}, {\bf x}) N({\bf y}) \td S_y
\end{equation}

\section{2D}

\begin{align}
I &= \lim_{\epsilon \to 0+} \int_{\Sigma-\Sigma(\epsilon)} G({\bf y}(\xi), {\bf x}) N(\xi) J(\xi) \td \xi \nonumber \\
&= \lim_{\epsilon \to 0+} \int_{\Sigma-\Sigma(\epsilon)} G({\bf y}(\xi), \xi_0) N(\xi) J(\xi) \td \xi \nonumber \\
&= \lim_{\epsilon \to 0+} \left\{ \int_{\alpha^+(\epsilon)}^{R^+} G^+(\rho) N^+(\rho) J^+(\rho) \td \rho +
\int_{\alpha^-(\epsilon)}^{R^-} G^-(\rho) N^-(\rho) J^-(\rho) \td \rho
\right\}
\end{align}

\begin{align}
G^+ N^+ J^+ = \frac{F_{-2}^+}{\rho^2} + \frac{F_{-1}^+}{\rho} + O(1) \nonumber \\
G^- N^- J^- = \frac{F_{-2}^-}{\rho^2} + \frac{F_{-1}^-}{\rho} + O(1)
\end{align}

\begin{multline}
I =
\int_{0}^{R^+} G^+ N^+ J^+ - \frac{F_{-2}^+}{\rho^2} - \frac{F_{-1}^+}{\rho} \td \rho +
\int_{0}^{R^-} G^- N^- J^- - \frac{F_{-2}^-}{\rho^2} - \frac{F_{-1}^-}{\rho} \td \rho \\
-
\left(
\frac{F_{-2}^+}{R^+}
+ \frac{F_{-2}^-}{R^-} 
\right)
+ F_{-1}^+ \ln R^+
+ F_{-1}^- \ln R^- \\
+
\lim_{\epsilon \to 0+}
\left\{
\frac{F_{-2}^+}{\alpha^+(\epsilon)}
+ \frac{F_{-2}^-}{\alpha^-(\epsilon)}
- \left(F_{-1}^+ \ln \left|\alpha^+(\epsilon) \right|
+ F_{-1}^- \ln \left|\alpha^-(\epsilon) \right|\right)
\right\}
\end{multline}

\begin{equation}
\alpha^+(\epsilon) =  \epsilon \beta^+ + \epsilon^2 \gamma^+ + O(\epsilon^3)
\end{equation}

\begin{multline}
I =
\int_{0}^{R^+} G^+ N^+ J^+ - \frac{F_{-2}^+}{\rho^2} - \frac{F_{-1}^+}{\rho} \td \rho +
\int_{0}^{R^-} G^- N^- J^- - \frac{F_{-2}^-}{\rho^2} - \frac{F_{-1}^-}{\rho} \td \rho \\
-
\left\{
F_{-2}^+\left(\frac{1}{R^+} + \frac{\gamma^+}{\beta^{+2}}\right)
+
F_{-2}^-\left(\frac{1}{R^-} + \frac{\gamma^-}{\beta^{-2}}\right)
\right\}
+ F_{-1}^+ \ln \left|\frac{R^+}{\beta^+}\right|
+ F_{-1}^- \ln \left|\frac{R^-}{\beta^-}\right| \\
+
\lim_{\epsilon \to 0+}
\frac{1}{\epsilon}
\left(
\frac{F_{-2}^+}{\beta^+}
+ \frac{F_{-2}^-}{\beta^-}
\right)
\end{multline}

\end{document}

