\documentclass[a4paper,11pt,twoside]{article}

\usepackage{palatino}
\usepackage{a4wide}
\usepackage{amsmath,bm}
\usepackage[utf8]{inputenc}
\usepackage[T1]{fontenc}

\newcommand{\td}{\mathrm{d}}
\newcommand{\te}{\mathrm{e}}

\title{Time Domain BEM}
\author{Peter Fiala}

\begin{document}

\maketitle

\tableofcontents

\section{Introduction}

Really just the first steps in TD BEM.

\section{Convolution integral}

The sound pressure $p$ at time $t$ and location ${\bf x}$ can be expressed using the convolutional form of the Helmholtz-integral:
%
\begin{equation}
\frac{1}{2}p({\bf x}, t) =
\int_{0}^{t}
\int_{S}
\left\{
G'_{n_y}({\bf x},{\bf y}, t-\tau) p({\bf y}, \tau)
-
G({\bf x}-{\bf y}, t-\tau) q({\bf y}, \tau)
\right\}
\td S_{y}
\td \tau
\end{equation}
%
where $G$ denotes the fundamental solution of the governing differential equation. For the case of the wave equation in 3D, the fundamental solution is expressed as
%
\begin{equation}
G({\bf x}-{\bf y}, t) = \frac{\delta(t-r/c)}{4\pi r}, \qquad r = |{\bf y}-{\bf x}|
\end{equation}

\section{Discretisation}

The pressure and its normal derivative are interpolated along the surface $S$ and time:
%
\begin{equation}
p({\bf x}, t) =
\sum_{\beta} \sum_j
p^{\beta}_{j} N_j({\bf x}) T_{\beta}(t)
\end{equation}
%
Introducing the interpolation in the convolutional integral yields the expression
%
\begin{multline}
\frac{1}{2}p({\bf x}, t) =
\sum_{\beta} \sum_j 
\int_{0}^{t} \int_{S}
G'_{n_y}({\bf x}-{\bf y}, t-\tau)
N_j({\bf y}) T_{\beta}(\tau)
\td S_{y} \td \tau
p^{\beta}_{j} \\
-
\sum_{\beta} \sum_j 
\int_{0}^{t} \int_{S}
G({\bf x}-{\bf y}, t-\tau)
N_j({\bf y}) T_{\beta}(\tau)
\td S_{y} \td \tau
q^{\beta}_{j} 
\end{multline}

\section{Collocation}

The integral equation is solved using the collocation method:
%
\begin{equation}
\frac{1}{2}p_{i}^{\alpha} =
\sum_{\beta = 1}^{\alpha} \sum_j 
H_{ij}^{\alpha \beta}
p^{\beta}_{j} \\
-
\sum_{\beta = 1}^{\alpha} \sum_j 
G_{ij}^{\alpha \beta}
q^{\beta}_{j} 
\end{equation}
%
where
%
\begin{align}
G_{ij}^{\alpha \beta} &=
\int_{0}^{t_{\alpha}} \int_{S}
G({\bf x}_i-{\bf y}, t_{\alpha}-\tau)
N_j({\bf y}) T_{\beta}(\tau)
\td S_{y} \td \tau
\nonumber \\
H_{ij}^{\alpha \beta} &=
\int_{0}^{t_{\alpha}} \int_{S}
G'_{n_y}({\bf x}_i-{\bf y}, t_{\alpha}-\tau)
N_j({\bf y}) T_{\beta}(\tau)
\td S_{y} \td \tau
\end{align}

Substituting the Green's function of the wave equation yields
%
\begin{align}
G_{ij}^{\alpha \beta}
&=
\int_{0}^{t_{\alpha}} \int_{S}
\frac{\delta(t_{\alpha}-\tau-r/c)}{4\pi r}
N_j({\bf y}) T_{\beta}(\tau)
\td S_{y} \td \tau
\nonumber \\
&=
\int_{S}
\frac{N_j({\bf y}) T_{\beta}(t_{\alpha}-r/c)}{4\pi r}
\td S_{y}
\end{align}
%
and
%
\begin{align}
H_{ij}^{\alpha \beta}
&=
\int_{0}^{t_{\alpha}} \int_{S}
\frac{\partial}{\partial r}\left(\frac{\delta(t_{\alpha}-\tau-r/c)}{4\pi r}\right)
r'_{n_{y}} N_j({\bf y}) T_{\beta}(\tau)
\td S_{y} \td \tau
\nonumber \\
&=
\int_{0}^{t_{\alpha}} \int_{S}
\left(
\frac{\delta'(t_{\alpha}-\tau-r/c)(-1/c)}{4\pi r}
-
\frac{\delta(t_{\alpha}-\tau-r/c)}{4\pi r^2}
\right)
r'_{n_{y}} N_j({\bf y}) T_{\beta}(\tau)
\td S_{y} \td \tau
\nonumber \\
&=
\int_{S}
\left(
\frac{T'_{\beta}(t_{\alpha}-r/c)/c}{4\pi r}
-
\frac{T_{\beta}(t_{\alpha}-r/c)}{4\pi r^2}
\right)
r'_{n_{y}} N_j({\bf y})
\td S_{y}
\end{align}
%

\section{Integral expressions}

\subsection{Linear time shape functions}

Introducing linear temporal interpolation functions defined with the equidistant base points at $\beta \Delta t$:
%
\begin{equation}
T_{\beta}(t) = \begin{cases}
\frac{t-(\beta-1)\Delta t}{\Delta t} & (\beta-1)\Delta t \le t < \beta \Delta t \\
1-\frac{t-\beta\Delta t}{\Delta t} & \beta\Delta t \le t < (\beta+1) \Delta t
\end{cases}, \qquad \beta = 1 \dots \alpha
\end{equation}
%
\begin{equation}
T'_{\beta}(t) = \begin{cases}
\frac{1}{\Delta t} & (\beta-1)\Delta t \le t < \beta \Delta t \\
-\frac{1}{\Delta t} & \beta\Delta t \le t < (\beta+1) \Delta t
\end{cases}, \qquad \beta = 1 \dots \alpha
\end{equation}
%
or, as used above:
%
\begin{equation}
T_{\alpha-\gamma}(t_{\alpha}-r/c)
= \begin{cases}
\gamma + 1 -r^* & \gamma < r^* \le \gamma+1 \\
\gamma + 1 +r^* & \gamma-1 < r^* \le \gamma
\end{cases} \nonumber \\
\end{equation}
%
\begin{equation}
T'_{\alpha-\gamma}(t_{\alpha}-r/c) = \begin{cases}
\frac{1}{\Delta t} & \gamma < r^* \le \gamma+1 \\
-\frac{1}{\Delta t} & \gamma-1 < r^* \le \gamma
\end{cases}
\end{equation}
%
where $r^* = r/c\Delta t$ is the normalised distance from the collocation point


\begin{align}
G_{ij}^{\gamma}
&=
\frac{\gamma + 1}{4\pi}
\int_{S_i^{\gamma}}
\frac{N_j({\bf y})}{r}
\td S_{y}
+
\frac{1}{4\pi c \Delta t}
\left(
\int_{S_i^{\gamma-}}
N_j({\bf y})
\td S_{y}
-
\int_{S_i^{\gamma+}}
N_j({\bf y})
\td S_{y}
\right) \nonumber \\
H_{ij}^{\gamma}
&=
-
\frac{\gamma+1}{4\pi}
\int_{S_{i}^{\gamma}}
\frac{r'_{n_{y}} N_j({\bf y})}{r^2}
\td S_{y}
+
\frac{1}{2\pi c \Delta t}
\left(
\int_{S_{i}^{\gamma+}}
\frac{r'_{n_{y}} N_j({\bf y})}{r}
\td S_{y}
-
\int_{S_{i}^{\gamma-}}
\frac{r'_{n_{y}} N_j({\bf y})}{r}
\td S_{y}
\right)
\end{align}

For constant triangles
%
\begin{align}
G_{ij}^{\gamma}
&=
\frac{\gamma + 1}{4\pi}
\int_{S_{ij}^{\gamma}}
\frac{1}{r}
\td S_{y}
+
\frac{S_{ij}^{\gamma-} - S_{ij}^{\gamma+}}{4\pi c \Delta t}
\nonumber \\
H_{ij}^{\gamma}
&=
-
\frac{\gamma+1}{4\pi}
\int_{S_{ij}^{\gamma}}
\frac{r'_{n_{y}}}{r^2}
\td S_{y}
+
\frac{1}{2\pi c \Delta t}
\left(
\int_{S_{ij}^{\gamma+}}
\frac{r'_{n_{y}}}{r}
\td S_{y}
-
\int_{S_{ij}^{\gamma-}}
\frac{r'_{n_{y}}}{r}
\td S_{y}
\right)
\end{align}



\end{document}

