\documentclass[a4paper,10pt,twocolumn]{article}

\usepackage{amsmath}
\newcommand{\td}{\mathrm{d}}
\newcommand{\te}{\mathrm{e}}

\begin{document}

\begin{align}
\int_0^1 f(x) \ln(x) \td x
& = \int_{\infty}^0 f(\te^{-t}) \ln(\te^{-t}) \te^{-t} (-\td t) \nonumber \\
& = \int_{0}^{\infty} -t f(\te^{-t}) \te^{-t} \td t
\end{align}


\section{Analytical singular integrals of the Poisson kernel}

\subsection{2D case}

\begin{equation}
G(r) = -\frac{\ln |r|}{2\pi}
\end{equation}

\subsubsection{Collocation over constant line}

\begin{equation}
\int_{-d_1}^{d_2} G(r) \td r
=
\frac{d - d_1 \ln |d_1| - d_2 \ln |d_2|}{2\pi}
\end{equation}


\subsubsection{Galerkin over constant line}

\begin{equation}
\int_{0}^{d} \int_{0}^{d} G(x-y) \td y \td x
=
d^2\frac{3-2\ln d}{4\pi}
\end{equation}


\subsection{3D case}

\begin{equation}
G(r) = \frac{1}{4\pi r}
\end{equation}

\subsubsection{Collocation over constant triangle}

$\bf x_0$ denotes the collocational point

\begin{equation}
\int_0^{2\pi} \int_0^{R(\theta)} \frac{r \td r \td \theta}{4\pi r}
=
\frac{1}{4\pi}\int_0^{2\pi} R(\theta) \td \theta
\end{equation}






\end{document}

